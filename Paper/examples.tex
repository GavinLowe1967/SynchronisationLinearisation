\section{Experiments}
\label{sec:experiments}

In this section we describe experiments based on our testing framework. 

%%%%%

\begin{figure}
\begin{center}
\begin{tabular}{lccc}
Category            & Arity & Stateful? & Heterogeneous? \\ \hline
Synchronous channel & 2     & N         & Y \\
Filter channel      & 2     & N         & Y \\
Men and women        & 2     & N         & Y \\
Exchanger           & 2     & N         & N \\
Counter channel     & 2     & Y         & Y \\
Two families        & 2     & Y         & Y \\
One family          & 2     & Y         & N \\
ABC                 & 3     & N         & Y \\
Barrier             & $n$   & N         & N \\
Enrollable barrier  & $1\mathord{..} n$, 1 & Y & N \\
Timeout channel     & 2, 1  & N         & Y \\
Timeout exchanger   & 2, 1  & N         & N \\
Closeable channel   & 2, 1  & Y         & Y \\
Terminating queue   & 1, $n$ & Y        & N  
\end{tabular}
\end{center}
\caption{Example interfaces of synchronisation objects.  \label{fig:examples}}
\end{figure}

% Channel with counter & 2    & Y         & Y \\
% ABC with counter    & 3     & Y         & Y \\
% Barrier with counter & $n$  & Y         & N \\   
% Add combining barrier?    

%%%%%

We consider synchronisation objects implementing a number of interfaces,
summarised in Figure~\ref{fig:examples}.  Most of the interfaces were
described in earlier sections (namely synchronous channel, filter channel,
exchanger, counter channel, barrier, enrollable barrier, timeout channel,
closeable channel, and terminating queue).
%
The \emph{men and women} problem involves two families of threads, known as
men and women: each thread wants to pair off with a thread of the other type;
each passes in its own identity, and expects to receive back the identity of
the thread with which it has paired.  
%
In the \emph{two families} problem,
there are two families of threads, with $n$~threads of each family; each
thread calls an operation~$n$ times, and each execution should synchronise
with a thread of the opposite family, a different thread each time.  In the
\emph{one family} problem, there are $n$~threads, each of which calls an
operation $n-1$~times, and each time should synchronise with a different
thread.  
%
The \emph{ABC} problem can be thought of as a ternary version of the men and
women problem: there are three types of threads, A, B and C; each
synchronisation involves one thread of each type.  
%
Finally, the \emph{timeout exchanger} is a timed version of the exchanger: if
a thread fails to exchange data with another thread, it can timeout and return
an appropriate result.

For each interface, we have implemented a tester using the direct algorithm
from Section~\ref{sec:direct}; for most, we have also implemented a tester
using the two-step approach from Section~\ref{sec:lin-testing}.

For each interface, we have produced a correct implementation.  For most
interfaces, we have also implemented one or more faulty versions that fail to
achieve either synchronisation linearisation or progressibility.  The faulty
versions mostly have realistic mistakes: a few are genuine bugs; others are
similar bugs we have seen from students.

%%%%%%%%%%%%%%%%%%%%%%%%%%%%%%%%%%%%%%%%%%%%%%%%%%%%%%%

\subsection{Experiments}

We describe various experiments below.  The purpose of testing is to find
bugs.  We therefore concentrate on the time taken to find bugs.  Questions we
want to answer include:
\begin{itemize}
\item Which works better, the direct algorithm or the two-step algorithm?  
\item How should we choose parameters (number of threads to run, number of
  iterations performed by each thread, etc.)\ for testing?  
\item Is this approach effective at finding bugs?
\end{itemize}

The experiments were performed on a dedicated eight-core machine (two 2.40GHz
Intel(R) Xeon(R) E5620 CPUs, with 12GB of RAM, but limited to 4GB of heap
space).

In each experiment below, we consider a synchronisation object with a bug that
causes a failure of synchronisation linearisation, but does not lead to a
deadlock.  We performed a number of \emph{runs} of a tester on the
synchronisation object.  In each run, a particular number of threads performed
a particular number of operation calls on the synchronisation object; the
relevant algorithm was then used to decide whether the log history was
synchronisation linearisable or two-step linearisable.

Each \emph{observation} performed multiple runs until an error was detected,
and recorded the time taken.  Each observation was performed as a separate
operating system process, with the aim of making observations independent,
avoiding dependencies caused by, for example, garbage collection, caching
behaviour, and just-in-time compilation.  Thus each observation was as close
as possible to a normal use case.

For each data point in the experiments, we performed 100 observations.  We
give the average running time and a 95\%-confidence interval for that average
(following~\cite{GBE2007}).  The number of observations is chosen so as to
obtain a reasonably small confidential interval, but avoiding excessively long
experiments.

%%%%%%%%%%%%%%%%%%%%%%%%%%%%%%%%%%%%%%%%%%%%%%%%%%%%%%%

We start with the second of the questions above, how to choose parameters for
testing.  Each of the graphs in Figures~\ref{fig:params-experiment-1}
and~\ref{fig:params-experiment-2} concerns a particular
tester.  Each data point represents a particular number $p$ of threads (given in
the key), and a particular number of operation executions by each thread
(given on the $X$-axis).
%, with 100 observations.  
The $Y$-axis gives the time in milliseconds.

%scala -cp .:/home/gavin/Scala/Util  experiments.BugParametersExperiment --doAll --samples 100
%% \documentclass[a4paper]{article}
%% \usepackage{pgfplots}
%% \pgfplotsset{compat=1.16}
%% \begin{document}


%\pgfplotsset{every semilogxaxis/.append style = { cycle list name=my mark list, }}

\def\gHeight{0.4579\textheight}
\def\gWidth{0.50\textwidth}
%% With labels on axes
%% \def\gHeight{0.44\textheight}
%% \def\gWidth{0.35\textwidth}

\begin{figure}
\begin{center}
\begin{tikzpicture}
\begin{semilogxaxis}[
  title = {Synchronous channel, direct},
 % Experiment on the time taken to find a bug using ChanTester --faulty,
  %% ylabel = Time (ms),
  %% xlabel = Number of iterations per thread,
  ymin = 0,
  log basis x=2,
  scaled ticks = false,
  legend pos = south east,
  height = \gHeight,
  width = \gWidth,
  cycle list name=my mark list,
]
\addplot+[error bars/.cd, y dir=both,y explicit] coordinates {
  (2,86.12661781999999) +- (0,1.7917221449438974)
  (4,80.95611876999999) +- (0,2.151870544775658)
  (8,79.33566509) +- (0,2.2847148738449485)
  (16,75.60885972) +- (0,2.077410210931587)
  (32,75.91483873) +- (0,1.77206458988127)
};
\addlegendentry{$p$ = 2}
\addplot+[error bars/.cd, y dir=both,y explicit] coordinates {
  (2,88.15634318000001) +- (0,2.280526062395564)
  (4,82.28905311) +- (0,2.3893457135505605)
  (8,81.42642638) +- (0,2.343502687069828)
  (16,81.70105392) +- (0,2.19568045053051)
  (32,83.99210701000001) +- (0,1.1896791019624882)
};
\addlegendentry{$p$ = 4}
\addplot+[error bars/.cd, y dir=both,y explicit] coordinates {
  (2,88.53399624) +- (0,2.5770165231292586)
  (4,86.50421789) +- (0,1.8202756998260026)
  (8,88.31893217) +- (0,1.0446761315476831)
  (16,92.69791773) +- (0,1.1625119157788526)
  (32,100.82756555) +- (0,1.026210111624912)
};
\addlegendentry{$p$ = 8}
\addplot+[error bars/.cd, y dir=both,y explicit] coordinates {
  (2,100.60871656) +- (0,3.421021184767031)
  (4,98.96857434) +- (0,1.5976825567761581)
  (8,102.43561928) +- (0,1.0967809472697325)
  (16,111.79526836) +- (0,1.331421208545199)
};
\addlegendentry{$p$ = 16}
\end{semilogxaxis}
\end{tikzpicture}
%%%%%%%%%%%
\hfill%
\begin{tikzpicture}
\begin{semilogxaxis}[
  title = {Synchronous channel, two-step},
% Experiment on the time taken to find a bug using ChanTwoStepTester --faulty,
  %% ylabel = Time (ms),
  %% xlabel = Number of iterations per thread,
  ymin = 0,
  log basis x=2,
  scaled ticks = false,
  legend pos = south east,
  height = \gHeight,
  width = \gWidth,
  cycle list name=my mark list,
]
\addplot+[error bars/.cd, y dir=both,y explicit] coordinates {
  (2,97.48732987000001) +- (0,2.537052448574137)
  (4,98.00178478) +- (0,0.7155620511380979)
  (8,102.59852943000001) +- (0,1.248428236383366)
  (16,110.18692125) +- (0,1.8779254833338999)
  (32,122.86119026) +- (0,3.687510258368647)
};
\addlegendentry{$p$ = 2}
\addplot+[error bars/.cd, y dir=both,y explicit] coordinates {
  (2,103.68578293) +- (0,0.9185773564133626)
  (4,107.72147869) +- (0,1.1262832580516602)
  (8,115.23526969) +- (0,1.558869648950144)
  (16,129.42642214) +- (0,2.890432405032626)
  (32,147.61696638) +- (0,3.3701479571174087)
};
\addlegendentry{$p$ = 4}
\addplot+[error bars/.cd, y dir=both,y explicit] coordinates {
  (2,115.59173774) +- (0,1.1624918696707782)
  (4,129.18186215) +- (0,3.0346089254020145)
  (8,149.27067307) +- (0,4.752304398606616)
  (16,184.42443716) +- (0,8.787981873362584)
  (32,224.80307553999998) +- (0,12.904969334685587)
};
\addlegendentry{$p$ = 8}
\end{semilogxaxis}
\end{tikzpicture}

%%%%%%%%%%%%%%%%%%%%%%%%%%%%%%%%%%%%%%%%%%%%%%%%%%%%%%%

\begin{tikzpicture}
\begin{semilogxaxis}[
  title = {ABC problem, direct},
% Experiment on the time taken to find a bug using ABCTester --faulty,
  %% ylabel = Time (ms),
  %% xlabel = Number of iterations per thread,
  ymin = 0, ymax = 12000,
  log basis x=2,
  scaled ticks = false,
  legend pos = north east,
  height = \gHeight,
  width = \gWidth,
  cycle list name=my mark list,
]
\addplot+[error bars/.cd, y dir=both,y explicit] coordinates {
  (2,63148.00490261) +- (0,21759.613507248898)
  (4,651.16099215) +- (0,172.9165504857434)
  (8,889.50264446) +- (0,253.67598564893999)
  (16,988.93800777) +- (0,331.2238136254922)
  (32,2531.2923463800003) +- (0,637.8365814648288)
};
\addlegendentry{$p$ = 6}
\addplot+[error bars/.cd, y dir=both,y explicit] coordinates {
  (2,8285.14449145) +- (0,2778.240180885241)
  (4,538.35292637) +- (0,69.7619752736345)
  (8,570.41879925) +- (0,72.51216157132927)
  (16,1236.46460153) +- (0,603.1643293373226)
};
\addlegendentry{$p$ = 9}
\addplot+[error bars/.cd, y dir=both,y explicit] coordinates {
  (2,2353.5410284299996) +- (0,556.6201735112219)
  (4,425.88017894) +- (0,43.38091364960696)
  (8,460.78079447000005) +- (0,55.267882094272984)
  (16,927.89629072) +- (0,310.78678210463266)
};
\addlegendentry{$p$ = 15}
\addplot+[error bars/.cd, y dir=both,y explicit] coordinates {
  (2,2086.92048954) +- (0,486.9798287963568)
  (4,457.24736308999996) +- (0,57.960262260725735)
  (8,815.22070533) +- (0,137.83521886164093)
};
\addlegendentry{$p$ = 24}
\end{semilogxaxis}
\end{tikzpicture}
%%%%%%
\hfill%
\begin{tikzpicture}
\begin{semilogxaxis}[
  title = {ABC problem, two-step},
% Experiment on the time taken to find a bug using ABCTwoStepTester --faulty,
  %% ylabel = Time (ms),
  %% xlabel = Number of iterations per thread,
  ymin = 0, ymax = 12000,
  log basis x=2,
  scaled ticks = false,
  legend pos = north east,
  height = \gHeight,
  width = \gWidth,
  cycle list name=my mark list,
]
\addplot+[error bars/.cd, y dir=both,y explicit] coordinates {
  (2,65758.65076251) +- (0,25549.72875338341)
  (4,1045.58266548) +- (0,329.4516890468716)
  (8,1046.70000612) +- (0,190.21621707113712)
  (16,2034.9508281300002) +- (0,663.4683619888534)
};
\addlegendentry{$p$ = 6}
\addplot+[error bars/.cd, y dir=both,y explicit] coordinates {
  (2,8272.18414539) +- (0,2218.9195185871336)
  (4,647.6533947300001) +- (0,91.52294872037436)
  (8,877.6609487100001) +- (0,115.20504848610265)
};
\addlegendentry{$p$ = 9}
\addplot+[error bars/.cd, y dir=both,y explicit] coordinates {
  (2,3214.8299765799998) +- (0,832.8638371203842)
  (4,614.13599756) +- (0,80.38135215280637)
  (8,1010.49079758) +- (0,131.23949278860303)
};
\addlegendentry{$p$ = 12}
\end{semilogxaxis}
\end{tikzpicture}
\end{center}
\caption{Effect of choices of parameters for testers for a synchronous channel
  and the ABC problem.}
\label{fig:params-experiment-1}
\end{figure}

%%%%%%%%%%%%%%%%%%%%%%%%%%%%%%%%%%%%%%%%%%%%%%%%%%%%%%%
%%%%%%%%%%%%%%%%%%%%%%%%%%%%%%%%%%%%%%%%%%%%%%%%%%%%%%%

\begin{figure}
\begin{center}
\begin{tikzpicture}
\begin{semilogxaxis}[
  title = {Barrier, direct},
% Experiment on the time taken to find a bug using BarrierTester --faulty4,
  %% ylabel = Time (ms),
  %% xlabel = Number of iterations per thread,
  ymin = 0,
  log basis x=2,
  scaled ticks = false,
  legend pos = south east,
  height = \gHeight,
  width = \gWidth,
  cycle list name=my mark list,
]
\addplot+[error bars/.cd, y dir=both,y explicit] coordinates {
  (2,79.78356759) +- (0,0.6753803761895055)
  (4,80.21149943) +- (0,0.7417077841043016)
  (8,82.15990262999999) +- (0,0.9721482587110002)
  (16,85.56849578) +- (0,1.3170026624624744)
  (32,89.00165894) +- (0,1.214991840184841)
};
\addlegendentry{$p$ = 2}
\addplot+[error bars/.cd, y dir=both,y explicit] coordinates {
  (2,80.69491434) +- (0,1.277121828349371)
  (4,79.11638876) +- (0,0.8541881324957225)
  (8,82.34575759) +- (0,0.6647658092993566)
  (16,90.96288109999999) +- (0,0.7946518374359616)
  (32,105.6103622) +- (0,0.9084773991480668)
};
\addlegendentry{$p$ = 4}
\addplot+[error bars/.cd, y dir=both,y explicit] coordinates {
  (2,87.37795215000001) +- (0,1.148427979634296)
  (4,91.80450587) +- (0,0.9659897545948385)
  (8,101.90022009) +- (0,1.0396303105810654)
  (16,114.59724041) +- (0,1.2234727441382758)
  (32,134.96979653) +- (0,1.3399123540734243)
};
\addlegendentry{$p$ = 8}
\addplot+[error bars/.cd, y dir=both,y explicit] coordinates {
  (2,103.75736121) +- (0,1.1028844359069854)
  (4,114.74564184) +- (0,1.065963299194866)
  (8,126.3824616) +- (0,1.2558203057125663)
  (16,146.69931995) +- (0,1.4325749994158248)
};
\addlegendentry{$p$ = 16}
\end{semilogxaxis}
\end{tikzpicture}
%%%%%
\hfill%
\begin{tikzpicture}
\begin{semilogxaxis}[
  title = {Barrier, two-step},
% Experiment on the time taken to find a bug using BarrierTwoStepTester --faulty4,
  %% ylabel = Time (ms),
  %% xlabel = Number of iterations per thread,
  ymin = 0,
  log basis x=2,
  scaled ticks = false,
  legend style={at={(0.95,0.5)}},
  %legend pos = south east,
  height = \gHeight,
  width = \gWidth,
  cycle list name=my mark list,
]
\addplot+[error bars/.cd, y dir=both,y explicit] coordinates {
  (2,104.52545672) +- (0,1.1384114955505373)
  (4,106.5277703) +- (0,0.8659686125977645)
  (8,110.02191739) +- (0,1.0478095390846915)
  (16,117.76376726000001) +- (0,1.4735785997288966)
  (32,126.09393337) +- (0,1.2584474029641417)
};
\addlegendentry{$p$ = 4}
\addplot+[error bars/.cd, y dir=both,y explicit] coordinates {
  (2,473.48318351) +- (0,6.665762120027659)
  (4,484.3246579) +- (0,7.588975675101476)
  (8,491.51942674000003) +- (0,5.436439641721073)
  (16,499.12677864999995) +- (0,6.026954379109569)
  (32,507.43132356) +- (0,7.060360684052092)
};
\addlegendentry{$p$ = 8}
\addplot+[error bars/.cd, y dir=both,y explicit] coordinates {
  (2,4723.944651850001) +- (0,88.7333271725884)
  (4,5140.44886425) +- (0,390.9244260810367)
  (8,4984.51105782) +- (0,299.05582122501505)
  (16,4846.063625010001) +- (0,185.3313893605648)
};
\addlegendentry{$p$ = 10}
\end{semilogxaxis}
\end{tikzpicture}

%%%%%%%%%%%%%%%%%%%%%%%%%%%%%%%%%%%%%%%%%%%%%%%%%%%%%%%

%% \begin{tikzpicture}
%% \begin{semilogxaxis}[
%%   title = {Exchanger, direct},
%% % Experiment on the time taken to find a bug using ExchangerTester --faulty,
%%   %% ylabel = Time (ms),
%%   %% xlabel = Number of iterations per thread,
%%   ymin = 0,
%%   log basis x=2,
%%   scaled ticks = false,
%%   legend pos = south east,
%%   height = \gHeight,
%%   width = \gWidth
%% ]
%% \addplot+[error bars/.cd, y dir=both,y explicit] coordinates {
%%   (2,79.4980078) +- (0,4.969728356062269)
%%   (4,81.5675652) +- (0,5.717326619474392)
%%   (8,77.60383034) +- (0,4.690777171745089)
%%   (16,81.10696336) +- (0,5.719550651715445)
%%   (32,81.79361565) +- (0,5.464883215270754)
%% };
%% \addlegendentry{$p$ = 4}
%% \addplot+[error bars/.cd, y dir=both,y explicit] coordinates {
%%   (2,73.68581234999999) +- (0,1.1772616810952101)
%%   (4,74.12068856) +- (0,2.2298505104315494)
%%   (8,74.53505454) +- (0,2.12662706415007)
%%   (16,74.04116777) +- (0,1.173710689795218)
%%   (32,76.37519547) +- (0,4.468775877914856)
%% };
%% \addlegendentry{$p$ = 8}
%% \addplot+[error bars/.cd, y dir=both,y explicit] coordinates {
%%   (2,83.19619601000001) +- (0,1.0357204067709476)
%%   (4,84.16564632) +- (0,1.0339471730559555)
%%   (8,83.61105594) +- (0,1.1567691820915948)
%%   (16,83.87681695) +- (0,1.1166611666195976)
%%   (32,83.29783656999999) +- (0,0.9106413255446043)
%% };
%% \addlegendentry{$p$ = 16}
%% \addplot+[error bars/.cd, y dir=both,y explicit] coordinates {
%%   (2,112.2743958) +- (0,2.4499531527325344)
%%   (4,112.05081016) +- (0,2.1363049579932554)
%%   (8,110.72104375) +- (0,2.154547825270566)
%%   (16,113.45533171) +- (0,2.3122404291087393)
%% };
%% \addlegendentry{$p$ = 32}
%% \end{semilogxaxis}
%% \end{tikzpicture}
%% %%%%%
%% \hfill%
%% \begin{tikzpicture}
%% \begin{semilogxaxis}[
%%   title = {Exchanger, two-step},
%% % Experiment on the time taken to find a bug using ExchangerTwoStepTester --faulty,
%%   %% ylabel = Time (ms),
%%   %% xlabel = Number of iterations per thread,
%%   ymin = 0,
%%   log basis x=2,
%%   scaled ticks = false,
%%   %legend pos = north east,
%%   legend style={at={(0.95,0.5)}},
%%   height = \gHeight,
%%   width = \gWidth
%% ]
%% \addplot+[error bars/.cd, y dir=both,y explicit] coordinates {
%%   (2,123.86405796) +- (0,8.037523435641969)
%%   (4,116.22122895) +- (0,5.163792850498938)
%%   (8,117.38155825) +- (0,4.1538189226791475)
%%   (16,127.48406541) +- (0,8.069867651568115)
%%   (32,123.89457534) +- (0,5.784136268257348)
%% };
%% \addlegendentry{$p$ = 4}
%% \addplot+[error bars/.cd, y dir=both,y explicit] coordinates {
%%   (2,535.28604669) +- (0,24.91982914948026)
%%   (4,489.79530002999996) +- (0,30.018925715872925)
%%   (8,501.71219262) +- (0,28.351449932028633)
%%   (16,483.71662317) +- (0,29.4417123713682)
%%   (32,482.93430472000006) +- (0,30.11208912390406)
%% };
%% \addlegendentry{$p$ = 8}
%% \addplot+[error bars/.cd, y dir=both,y explicit] coordinates {
%%   (2,4964.56554911) +- (0,522.4975339054702)
%%   (4,4489.6878968500005) +- (0,489.42545179368557)
%%   (8,4783.7948075) +- (0,482.77575329662307)
%%   (16,4696.5925643) +- (0,541.8583452036177)
%%   (32,5075.38201705) +- (0,513.4696344347326)
%% };
%% \addlegendentry{$p$ = 10}
%% \end{semilogxaxis}
%% \end{tikzpicture}
%% \end{center}
%% \caption{Effect of choices of parameters for testers for a barrier and an
%%   exchanger.}
%% \label{fig:params-experiment-2}
%% \end{figure}

%%%%%%%%%%%%%%%%%%%%%%%%%%%%%%%%%%%%%%%%%%%%%%%%%%%%%%%
%%%%%%%%%%%%%%%%%%%%%%%%%%%%%%%%%%%%%%%%%%%%%%%%%%%%%%%

%% \begin{figure}
%% \begin{center}
\begin{tikzpicture}
\begin{semilogxaxis}[
  title = {Closeable channel, direct},
% Experiment on the time taken to find a bug using CloseableChanTester --faultyWrapped,
  %% ylabel = Time (ms),
  %% xlabel = Number of iterations per thread,
  ymin = 0,
  log basis x=2,
  scaled ticks = false,
  legend pos = north east,
  height = \gHeight,
  width = \gWidth,
  cycle list name=my mark list,
]
\addplot+[error bars/.cd, y dir=both,y explicit] coordinates {
  (2,217.87036477) +- (0,20.95342905917427)
  (4,230.86778241) +- (0,19.798701188208636)
  (8,261.64808591999997) +- (0,22.670538051657655)
  (16,260.33807355) +- (0,21.011043536506982)
  (32,283.35029041) +- (0,24.207343654890213)
};
\addlegendentry{$p$ = 2}
\addplot+[error bars/.cd, y dir=both,y explicit] coordinates {
  (2,211.16205488999998) +- (0,11.940481774546184)
  (4,201.49129182) +- (0,11.327761520940962)
  (8,209.65499441) +- (0,11.493818434761115)
  (16,227.03573612) +- (0,13.275939252852606)
  (32,245.23078971) +- (0,12.431403656046855)
};
\addlegendentry{$p$ = 4}
\addplot+[error bars/.cd, y dir=both,y explicit] coordinates {
  (2,229.76947737) +- (0,13.333998715056868)
  (4,205.40328603) +- (0,7.113226630181317)
  (8,215.71346077) +- (0,8.57622815594333)
  (16,225.81513956) +- (0,9.079828437001673)
  (32,269.27598687) +- (0,13.53010438846008)
};
\addlegendentry{$p$ = 8}
\addplot+[error bars/.cd, y dir=both,y explicit] coordinates {
  (2,411.6651253) +- (0,67.13326503788613)
  (4,567.7329529) +- (0,216.01117553300745)
  (8,495.66493997000003) +- (0,104.46284260122805)
  (16,556.55072212) +- (0,134.35063241398308)
};
\addlegendentry{$p$ = 16}
\end{semilogxaxis}
\end{tikzpicture}
%%%%%%%%%%%%%%%%%%%%%%%%%
\hfill%
\begin{tikzpicture}
\begin{semilogxaxis}[
  title = {Closeable channel, two-step},
%Experiment on the time taken to find a bug using CloseableChannelTwoStepTester --faultyWrapped,
  %% ylabel = Time (ms),
  %% xlabel = Number of iterations per thread,
  ymin = 0, ymax = 700,
  log basis x=2,
  scaled ticks = false,
  legend pos = north east,
  height = \gHeight,
  width = \gWidth,
  cycle list name=my mark list,
]
\addplot+[error bars/.cd, y dir=both,y explicit] coordinates {
  (2,219.9772513) +- (0,15.934906768096278)
  (4,209.06976327) +- (0,15.358697413263384)
  (8,214.01946236) +- (0,15.628095370426285)
  (16,261.98785987) +- (0,18.4209251675354)
  (32,258.08770722) +- (0,18.753502541749246)
};
\addlegendentry{$p$ = 2}
\addplot+[error bars/.cd, y dir=both,y explicit] coordinates {
  (2,189.20459996) +- (0,8.697663071904486)
  (4,172.27710093000002) +- (0,9.278557282974742)
  (8,157.14860153) +- (0,6.995509080790417)
  (16,169.63906597) +- (0,8.24320928585933)
  (32,186.89724094) +- (0,8.402451799671582)
};
\addlegendentry{$p$ = 4}
\addplot+[error bars/.cd, y dir=both,y explicit] coordinates {
  (2,193.08478975999998) +- (0,9.818052840094685)
  (4,185.015977) +- (0,7.952847014557617)
  (8,190.04441344) +- (0,7.65325479343285)
  (16,211.562561) +- (0,11.461758427701987)
};
\addlegendentry{$p$ = 8}
\addplot+[error bars/.cd, y dir=both,y explicit] coordinates {
  (2,219.47016954) +- (0,23.733078040857574)
  (4,248.4342393) +- (0,42.645725442202924)
  (8,435.72753947) +- (0,333.66702849081315)
};
\addlegendentry{$p$ = 10}
\end{semilogxaxis}
\end{tikzpicture}
\end{center}
\caption{Effect of choices of parameters for testers for a barrier and a
  closeable channel.}
\label{fig:params-experiment-2}
\end{figure}
%% \end{document}


The experiments suggest that both types of tester work best with a fairly
small number of threads, each performing a fairly small number of operations
per run.  Some bugs are exhibited only when the number of threads exceeds the
arity of a synchronisation: for the ABC problem, two different
synchronisations interfere; for the closeable channel, the closing of the
channel interferes with a synchronisation.  We therefore recommend including
enough threads to find such bugs. 

%%  We therefore subsequently run four threads for binary
%% synchronisation objects, or six for the ABC problem (where the tester assumes
%% the number of threads is divisible by~three).  A barrier is of a different
%% nature, since it synchronises all the threads; but we again subsequently run
%% four threads on barriers.  In each of these cases, we arrange for each thread
%% to execute four operations.  An exception is the exchanger: as explained
%% earlier, we arrange for each thread to perform a single 

%%%%%%%%%%

These results suggest that the direct algorithms scale better than the
two-step approach as the number of threads increases.
Figure~\ref{fig:scalingExperiment} investigates this further.  Each graph
considers one type of synchronisation object, with the two plots representing
the two testers.  Each data point considers a particular number of threads
(given on the $X$-axis).
%, with 100 observations.  
The $Y$-axis again gives the time in milliseconds.  We omit results for the
two-step tester where testing sometimes ran out of memory.
%
The results confirm that the two-step approach does not scale well with the
number of threads.  In each case, the running time explodes at a particular
point, and tests often fail.  The running time also becomes erratic (as
indicated by the wide confidence intervals): some runs take an extremely long
time.  By contrast, the direct algorithms scale well.

% scala -cp .:/home/gavin/Scala/Util  experiments.BugParametersExperiment --scalingExperiments --samples 100
%% \documentclass[a4paper]{article}
%% \usepackage{pgfplots}
%% \pgfplotsset{compat=1.16}
%% \begin{document}

\begin{figure}
\begin{center}
\begin{tikzpicture}
\begin{axis}[
  title = Synchronous channel,
% Experiment on the time taken to find a bug using ABCTester --faulty,
  %% ylabel = Time (ms),
  %% xlabel = Number of iterations per thread,
  ymin = 0, ymax = 4000,
  log basis x=2,
  scaled ticks = false,
  legend pos = north west,
  height = \gHeight,
  width = \gWidth,
  %% xlabel = Number of threads
]
\addplot+[error bars/.cd, y dir=both,y explicit] coordinates {
  (2,80.17557097) +- (0,2.313976246200295)
  (4,82.28302028) +- (0,2.4335327027022076)
  (8,85.36077082999999) +- (0,1.6876257801962267)
  (12,92.07562671) +- (0,1.2105274136407866)
  (16,99.44679917) +- (0,1.541428322475591)
  (20,104.17721047) +- (0,1.1526221977538424)
  %% (22,108.18462238) +- (0,1.3651851641225)
  (24,120.06519478) +- (0,2.049365255112062)
  (28,131.57102554) +- (0,2.098694031220748)
  (32,137.73540709) +- (0,1.8049509031073891)
};
\addlegendentry{Direct}
\addplot+[error bars/.cd, y dir=both,y explicit] coordinates {
  (2,99.35065917) +- (0,2.290820993958403)
  (4,107.21510358) +- (0,1.08627145477178)
  (8,128.75744427) +- (0,3.02110024002185)
  (12,192.25562924000002) +- (0,18.038345750964854)
  (16,406.23790026999995) +- (0,98.63111643854575)
  (20,3704.92351963) +- (0,3686.316989775613)
};
\addlegendentry{Two-step}
\end{axis}
\end{tikzpicture}
%%%%%%%%%%
\begin{tikzpicture}
\begin{axis}[
  title = ABC,
  %% ylabel = Time (ms),
  %% xlabel = Number of iterations per thread,
  ymin = 0, ymax = 6500,
  log basis x=2,
  scaled ticks = false,
  legend pos = north west,
  height = \gHeight,
  width = \gWidth,
  %% xlabel = Number of threads
]
\addplot+[error bars/.cd, y dir=both,y explicit] coordinates {
  (6,738.97523388) +- (0,192.30437419901662)
  (9,516.3166476499999) +- (0,73.07929020702144)
  (12,428.05922672) +- (0,51.84354705496431)
  (18,437.17449813999997) +- (0,43.68155185436697)
  (21,468.40309231) +- (0,49.17296231702148)
  (24,480.95971837) +- (0,45.26460652255709)
  (30,503.82373912) +- (0,47.435795199862945)
};
\addlegendentry{Direct}
\addplot+[error bars/.cd, y dir=both,y explicit] coordinates {
  (6,1185.0105184000001) +- (0,500.305829959669)
  (9,649.7943049600001) +- (0,92.47896644220319)
  (12,523.2936806700001) +- (0,55.589694380875514)
  (18,1389.68275199) +- (0,265.9151307866405)
  (21,6136.98969674) +- (0,2374.586589732776)
};
\addlegendentry{Two-step}
\end{axis}
\end{tikzpicture}

\bigskip

%%%%%%%%%%%%%%%%%%%%%%%%%%%%%%%%%%%%%%%%%%%%%%%%%%%%%%%

\begin{tikzpicture}
\begin{axis}[
  title = Exchanger,
  %% ylabel = Time (ms),
  %% xlabel = Number of iterations per thread,
  ymin = 0, ymax = 5000,
  log basis x=2,
  scaled ticks = false,
  legend pos = north east,
  height = \gHeight,
  width = \gWidth,
  %% xlabel = Number of threads
]
\addplot+[error bars/.cd, y dir=both,y explicit] coordinates {
  (4,83.98867041) +- (0,5.84972087508774)
  (6,71.83952737) +- (0,2.065354913851141)
  (8,74.11781601) +- (0,3.1537908684816114)
  (10,75.0670475) +- (0,0.805666269953274)
  (12,77.90637123) +- (0,0.9281723987237208)
  (16,84.82022478) +- (0,0.9538758094074133)
  (20,88.69409691) +- (0,1.274185258318204)
  (24,94.56927361) +- (0,1.0908692638240356)
  (28,100.10252856999999) +- (0,1.251472815008965)
  (32,111.07823459999999) +- (0,2.241102402077327)
};
\addlegendentry{Direct}
\addplot+[error bars/.cd, y dir=both,y explicit] coordinates {
  (4,118.93380857) +- (0,5.801655675442106)
  (6,165.84577971000002) +- (0,4.284111269787625)
  (8,512.19757177) +- (0,24.459417584181057)
  (10,4367.849069819999) +- (0,520.9152639377908)
};
\addlegendentry{Two-step}
\end{axis}
\end{tikzpicture}
%%%%%%%%%%
\begin{tikzpicture}
\begin{axis}[
  title = Barrier,
  %% ylabel = Time (ms),
  %% xlabel = Number of iterations per thread,
  ymin = 0, ymax = 5000,
  log basis x=2,
  scaled ticks = false,
  legend pos = north east,
  height = \gHeight,
  width = \gWidth,
  %% xlabel = Number of threads
]
\addplot+[error bars/.cd, y dir=both,y explicit] coordinates {
  (4,79.15746188) +- (0,0.8148235154891171)
  (6,86.35219253) +- (0,0.9117832455251536)
  (8,91.54211674) +- (0,0.8558335206240062)
  (10,97.45011204000001) +- (0,0.9875873719463663)
  (12,103.31072401) +- (0,1.0294361811920485)
  (16,114.19281606) +- (0,1.1061892417043115)
  (24,134.09490376) +- (0,1.5693232175285319)
  (32,167.9751856) +- (0,2.0879807791854277)
};
\addlegendentry{Direct}
\addplot+[error bars/.cd, y dir=both,y explicit] coordinates {
  (4,106.80990786) +- (0,1.028433378278398)
  (6,165.23789938) +- (0,1.4234599346997459)
  (8,482.3549788) +- (0,5.198066954797921)
  (10,4774.87041949) +- (0,180.19593219438985)
};
\addlegendentry{Two-step}
\end{axis}
\end{tikzpicture}
\end{center}
\caption{The effect of increasing the number of threads on the time to find
  bugs.}
\label{fig:scalingExperiment}
\end{figure}
%% \begin{tikzpicture}
%% \begin{axis}[
%%   title = Experiment on the time taken to find a bug using ABCTester --faulty,
%%   %% ylabel = Time (ms),
%%   %% xlabel = Number of iterations per thread,
%%   ymin = 0,
%%   log basis x=2,
%%   scaled ticks = false,
%%   legend pos = north east,
%%   height = 0.8\textheight,
%%   width = 0.6\textwidth,
%%   %% xlabel = Number of threads
%% ]
%% \addplot+[error bars/.cd, y dir=both,y explicit] coordinates {
%%   (4,96.63427718999999) +- (0,2.5163477297936216)
%%   (8,113.01763845) +- (0,9.94350003552628)
%%   (12,118.24920412) +- (0,3.8038559172848885)
%%   (16,127.87080176) +- (0,2.1334251628644845)
%%   (18,137.21550991) +- (0,5.109093758472966)
%%   (20,148.60901783) +- (0,4.363024022806203)
%%   (24,169.97987766) +- (0,6.020211068562741)
%%   (28,176.99855642) +- (0,4.140373056156611)
%%   (32,188.88253531) +- (0,6.00941825984061)
%% };
%% \addlegendentry{ChanCounterTester}
%% \addplot+[error bars/.cd, y dir=both,y explicit] coordinates {
%%   (4,107.01148408) +- (0,1.3175787272172261)
%%   (8,135.87752678) +- (0,2.7028257297729663)
%%   (12,283.46283038) +- (0,23.433002506513333)
%%   (16,1254.59913635) +- (0,443.72455626905014)
%% };
%% \addlegendentry{ChanCounterTwoStepTester}
%% \end{axis}
%% \end{tikzpicture}
%% %\end{document}



We believe the reason the two-step approach scales poorly is as follows.  The
linearisation tester tries to find a linearisation order for operation
executions, via a depth-first search.  At each step, it picks a particular
operation execution to try to linearise next.  If it picks wrongly, it might
have to consider many nodes of the search graph before backtracking.  This can
be the case with two-step testing, because if it picks the wrong $\op_1$ to try
to linearise, it will only discover this fact when it reaches the
corresponding $\overline{\op}_1$ and finds the wrong value is returned.  This
latter event might be much later in the history.  To reach it, the tester has
to consider many possibilities for ordering other operation executions.

%%%%%%%%%%

Figure~\ref{fig:bugFindingExperiment} gives times to find various bugs with
the two testers.  Based on the earlier experiments, in most cases we ran four
threads, each executing four operations; for the ABC testers, we ran six
threads, each executing four operations; for the (untimed) exchanger, we ran
eight threads, each performing one operation; for the two-families object, we
ran four threads (two from each family), each performing two operations; and
for the one-family object, we ran four threads, each performing three
operations.  The table gives average times in milliseconds to detect the bug,
with 95\%-confidence intervals.

%%%%%

\begin{figure}
%scala -cp .:/home/gavin/Scala/Util  experiments.BugFinderExperiment --samples 100
\begin{tabular}{lr@{$\,\pm\,$}lr@{$\,\pm\,$}l}
Synchronisation object & \multicolumn{2}{c}{Direct} &
\multicolumn{2}{c}{Two-step} \\ \hline
Synchronous channel  &	85 	 & 3 &  	109	 & 2 \\
Filter channel & \multicolumn{2}{c}{---} & \multicolumn{2}{c}{---} \\
Men and women  &	75	 & 1 &  	108	 & 4 \\
Exchanger  	&       73	 & 1 &  	511	 & 25 \\
Counter channel &  	94	 & 2 &  	106	 & 1 \\
Two families &  	299	 & 34 &  	267	 & 25 \\
One family &    	336	 & 27 & \multicolumn{2}{c}{---} \\
ABC &    	        717	 & 225 &  	928	 & 216 \\
Barrier &       	80	 & 1 &   	106	 & 1 \\
Enrollable barrier &  	132	 & 5 &  	149	 & 6 \\
Timeout channel  &	121	 & 5 &  	135	 & 5 \\
Timeout exchanger & 	288	 & 64 &  	231	 & 18 \\
Closeable channel & 	191	 & 11 &  	173	 & 9 \\
Terminating queue & 	103	 & 1 &  	105	 & 1
\end{tabular}

\framebox{???} Add one family, filter channel? % Remove counter channel? 
\caption{Times to find bugs.}
\label{fig:bugFindingExperiment}
\end{figure}

%%%%%

All the testers work well, with the average time to find each bug below one
second.  Of course, other bugs might be harder to find, because they are
triggered on fewer runs.  However, our results do suggest that our techniques
are effective at finding most bugs.

In most cases, the direct tester is faster than the two-step one; and the
two-step tester is never significantly faster.  We therefore recommend the
direct algorithms.  This approach has an additional advantage: our experience
is that it is easier to create the testing program based on these algorithms,
whereas using the two-step approach involves designing and encoding the
appropriate automaton, which can be somewhat tricky. 

We now consider synchronisation progressibility.  Our experience is that if a
synchronisation object does not satisfy progressibility, then this can lead to
a total deadlock.  Thus, in most cases, testing for synchronisation
linearisability will also detect progressibility bugs.  However, this is no
guarantee.    

%% We also carried out some experiments to assess the throughput on correct
%% implementations when testing for progressibility.  However, the times were
%% dominated by the times waiting for timeouts.  Where there were differences
%% between examples, these simply reflect the probability of the system
%% completing on its own, and not having to wait for the timeout.  We omit the
%% results, because they are uninteresting.

\framebox{Add experiments?}
