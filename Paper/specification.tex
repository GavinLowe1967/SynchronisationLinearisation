\section{Specifying synchronisations}
\label{sec:spec}

In this section we describe how synchronisations can be formally specified.
We start by considering \emph{heterogeneous binary} synchronisation,
i.e.~where every synchronisation is between executions of \emph{two different}
operations.  We allow stateful synchronisation objects (which includes
stateless objects as degenerate cases).  We generalise in
Section~\ref{ssec:spec-variations}. 

For the moment, we assume that the synchronisation object has two operations,
each of which has a single parameter, as follows.
%
\begin{scala}
def op£\s1£(x£\s1£: A£\s1£): B£\s1£
def op£\s2£(x£\s2£: A£\s2£): B£\s2£
\end{scala}
%
(We can model a concrete operation that takes $k \ne 1$ parameters by an
operation that takes a $k$-tuple as its parameter.  We identify a 0-tuple with
the unit value, but will sometimes omit that value in examples.)
%
In addition, the synchronisation object might have some state.
Each execution of~|op|\s1 must synchronise with an execution of~|op|\s2, and
vice versa.  The result of each execution may depend on the two parameters,
|x|$_1$ and |x|$_2$, and the current state.  In addition, the state may be
updated.  The external behaviour is consistent with the synchronisation
happening atomically at some point within the duration of both operation
executions (which implies that the executions must overlap): we refer to this
point as the \emph{synchronisation point}.

Synchronisation linearisation is defined in terms of a \emph{synchronisation
  specification object}: we define these specification objects in the next
subsection.  In Section~\ref{sec:specification-linearisability}, we review the
notion of linearisation, on which synchronisation linearisation is based.  We
then define synchronisation linearisation for binary heterogeneous
synchronisation objects in Section~\ref{sec:sync-lin}.  We generalise to other
classes of synchronisation objects in Section~\ref{ssec:spec-variations}.  In
Section~\ref{sec:locality}, we show that our definition satisfies
\emph{locality}: a collection of objects satisfies synchronisation
linearisability if and only if each individual object does.   
%% We
%% present our liveness property, synchronisation progressibility, in
%% Section~\ref{sec:progress}.

%%%%%%%%%%%%%%%%%%%%%%%%%%%%%%%%%%%%%%%%%%%%%%%%%%%%%%%

\subsection{Synchronisation specification objects}

Each synchronisation object, with a signature as above, can be specified using
a \emph{synchronisation specification object} with the following signature.
%
\begin{scala}
class Spec{
  def sync(x£\s1£: A£\s1£, x£\s2£: A£\s2£): (B£\s1£, B£\s2£)
}
\end{scala}
%
The idea is that if two executions |op|\s1|(x|\s1|)| and |op|\s2|(x|\s2|)|
synchronise, then the results |y|\s1 and |y|\s2 of the executions are such
that $\sm{sync}(\sm x_1, \sm x_2)$ returns the pair |(y|\s1|, y|\s2|)|.
%% (We allow |sync| to be nondeterministic; but in all our examples it will be
%% deterministic.)  
The specification object might have private state, which can be accessed and
updated within~|sync|.  Note that executions of |sync| occur
\emph{sequentially}.  

Informally, the synchronisation specification object can be seen as an
idealised description of the effects of the synchronisation, as if---instead
of calling $\op_1$ and~$\op_2$---the two threads had jointly called |sync|,
each supplying one parameter, and each taking one component of the result.

In general, |sync| could be nondeterministic, and so allow several different
results.  However, in all our examples, |sync| will be deterministic; and we
will require it to be deterministic when we consider testing, from
Section~\ref{sec:testing-hacking} onwards.

%% We assume that |sync| is a deterministic function of its parameters and the
%% state.

We formalise below what it means for a synchronisation object to satisfy the
requirements of a synchronisation specification object.  But first, we give
some examples to illustrate the style of specification. 

A generic definition of a specification object might take the following form: 
%
\pagebreak[3]
\begin{scala}
class Spec{
  private var state: S
  def sync(x£\s1£: A£\s1£, x£\s2£: A£\s2£): (B£\s1£, B£\s2£) = {
    require(guard(x£\s1£, x£\s2£, state))
    val res£\s1£ = f£\s1£(x£\s1£, x£\s2£, state); val res£\s2£ = f£\s2£(x£\s1£, x£\s2£, state)
    state = update(x£\s1£, x£\s2£, state)
    (res£\s1£, res£\s2£)
  }
}
\end{scala}
%
The object has some local state, which persists between executions.  The
|require| clause of |sync| specifies a precondition for the synchronisation to
take place: that precondition is described by the boolean function~|guard|.
The values |res|\s1 and |res|\s2 represent the results that should be returned
by the corresponding executions of~|op|\s1 and~|op|\s2, respectively.  The
function |update| describes how the local state should be updated.
%%  We assume the specification object is deterministic: |f|$_1$, |f|$_2$
%% and |update| are functions. 

For example, consider a synchronous channel with operations
\begin{scala}
def send(x: A): Unit
def receive(u: Unit): A
\end{scala}
%
(Note that we model the |receive| operation as taking a parameter of type
|Unit|, in order to fit our uniform setting.) 
%
This can be specified using a synchronisation specification object
with empty state:
%
\begin{scala}
class SyncChanSpec[A]{
  def sync(x: A, u: Unit): (Unit, A) = ((), x)
}
\end{scala}
%
If |send(x)| synchronises with |receive(())|, then the former receives the
unit value~|()|, and the latter receives~|x|. 

As another example, consider a filtering channel.
\begin{scala}
class FilterChan[A]{
  def send(x: A): Unit
  def receive(p: A => Boolean): A
}
\end{scala}
%
Here the |receive| operation is passed a predicate~|p| describing a required
property of any value received.  This can be specified using a stateless
specification object with operation
%
\begin{scala}
def sync(x: A, p: A => Boolean): (Unit, A) = { require(p(x)); ((), x) }
\end{scala}
%
The |require| clause specifies that executions |send(x)| and |receive(p)| can
synchronise only if |p(x)|.

As an example illustrating the use of state in the synchronisation object,
recall the synchronous channel with a sequence counter, |SyncChanCounter|,
from the introduction.  This can be specified using the following
specification object.
%
\begin{scala}
class SyncChanCounterSpec[A]{
  private var counter = 0
  def sync(x: A, u: Unit): (Int, (A, Int)) = {
    counter += 1; (counter, (x, counter))
  }
}
\end{scala}
%
Each synchronisation increments the counter, and the new value is returned to
each thread.

%%%%%%%%%%

\subsection{Linearisation}
\label{sec:specification-linearisability}

We formalise the allowable behaviours captured by a particular synchronisation
specification object.  Our definition has much in common with the well known
notion of \emph{linearisation}~\cite{herlihy-wing}, used for specifying
concurrent datatypes: with linearisation, operation invocations appear to
happen in a one-at-a-time order; for a synchronisation object, we want to
capture that synchronisations appear to happen in a one-at-a-time order.  We
start by reviewing linearisation.  There are a number of equivalent ways of
defining it: we choose a way that will be convenient subsequently.

A \emph{concurrent history} of an object~$o$ (either a concurrent datatype or
a synchronisation object) records the calls and returns of operations on~$o$.
It is a sequence of events of the following forms:
%
\begin{itemize}
\item $\call.op^i(x)$, representing a call of operation~$op$ with
  parameter~$x$;
\item $\return.op^i \:: y$, representing a return of an execution of~$op$,
  giving result~$y$.
\end{itemize}
%
Here $i$ is an \emph{execution identity}, used to identify a particular
execution, and to link the $\call$ and corresponding~$\return$.  In order to
be well formed, each execution identity must appear on at most one $\call$
event and at most one $\return$ event; and for each event $\return.op^i\::y$,
the history must contain an earlier event $\call.op^i(x)$, i.e.~for the same
operation and execution identity.  We consider only well formed histories
from now on.  

We say that a $\call$ event and a $\return$ event \emph{match}
if they have the same execution identifier.  A concurrent history is
\emph{complete} if for every $\call$ event, there is a matching $\return$
event, i.e.~no execution is still pending at the end of the history.

For example, consider the following complete concurrent history of a
concurrent datatype that is intended to implement a queue, with operations
|enq| and~|deq|.
%
\begin{eqnarray*}
h & = & 
  \seq{\begin{align}
    \call.\sm{enq}^1(5),\; \call.\sm{enq}^2(4),\; \call.\sm{deq}^3(), \\
    \return.\sm{enq}^1\::(),\; \return.\sm{deq}^3\::4,\;
    \return.\sm{enq}^2\::() }.
    \end{align}
\end{eqnarray*}
%
This history is illustrated by the timeline in Figure~\ref{fig:lin-timeline}.

%%%%%

\begin{figure}
\unScalaMid
%\def\X{node{$\cross$}}
\begin{center}
\begin{tikzpicture}[xscale = 0.9]
\draw[|-|] (0,0) -- node[above] {$\sm{enq}^1(5)\::()$} (3.5,0);
\draw (2.5,0) \X;
\draw[|-|] (1,-1) -- node[above] {$\sm{enq}^2(4)\::()$} (6,-1);
\draw (2,-1) \X;
\draw[|-|] (3,-2) -- node[above] {$\sm{deq}^3()\::4$} (5,-2);
\draw (4,-2) \X;
\end{tikzpicture}
\end{center}
\caption{Timeline representing the linearisation example.  Time runs from left
  to right; each horizontal line represents an operation execution, with the
  left-hand end representing the $\call$ event, and the right-hand end
  representing the $\return$ event.}
\label{fig:lin-timeline}
\scalaMid
\end{figure}

%%%%%

Linearisation is specified with respect to a linearisation specification
object~$Spec$, with the same operations (and signatures) as the concurrent
datatype in question.  A history of the specification object is a sequence of
events of the form:
%
\begin{itemize}
\item $op^i(x)\::y$ representing an execution of operation~$op$ with
  parameter~$x$, returning result~$y$; again $i$~is an execution identity,
  which must appear at most once in the history.
\end{itemize}
%
A history is \emph{legal} if it is consistent with the definition of~$Spec$,
i.e.~for each operation execution, the precondition is satisfied, and the
return value is as for the definition of the operation in~$Spec$.  In general,
the specification object could be nondeterministic, and so allow several
values that could be returned by an operation execution (although in all our
examples it will be deterministic).

%% We assume that the
%% specification object is deterministic: after a particular history, there is a
%% unique value that can be returned by each execution.

For example, consider the history
\begin{eqnarray*}
h_s & = & \seq{\sm{enq}^2(4)\::(),\; \sm{enq}^1(5)\::(),\; \sm{deq}^3()\::4}.
\end{eqnarray*}
%
This is a legal history for a specification object that represents a queue.
This history is illustrated by the ``$\cross$''s in
Figure~\ref{fig:lin-timeline}.

Let $h$ be a complete concurrent history, and let $h_s$ be a legal history of
the specification object~$Spec$.  We say that $h$ and~$h_s$ \emph{correspond}
if they contain the same executions, i.e., for each $\call.op^i(x)$ and
$\return.op^i\::y$ in $h$,\, $h_s$ contains $op^i(x)\::y$, and vice versa.  We
say that $h_s$ is a \emph{linearisation} of~$h$ if there is some way of
interleaving the two histories (i.e.~creating a history containing the events
of~$h$ and~$h_s$, preserving the order of events) such that each $op^i(x)\::y$
occurs between $\call.op^i(x)$ and $\return.op^i\::y$.  Informally, this
indicates that the executions of~$h$ appeared to take place in the order
described by~$h_s$, and that this order is legal according to the
specification object.  We say that $h$ is \emph{linearisable} with respect
to~$Spec$ in this case.

Continuing the running example, $h_s$ is a linearisation of~$h$, as evidenced
by the interleaving
\[
\seq{\begin{align}
  \call.\sm{enq}^1(5),\; \call.\sm{enq}^2(4),\; 
  \sm{enq}^2(4)\::(),\; \sm{enq}^1(5)\::(),\;   \call.\sm{deq}^3(), \\
  \return.\sm{enq}^1\::(),\; \sm{deq}^3\::4,\; 
  \return.\sm{deq}^3\::4,\; \return.\sm{enq}^2\::() },
  \end{align}
\]
as illustrated in  Figure~\ref{fig:lin-timeline}.
The points at which the events of~$h_s$ are inserted into~$h$ can be thought
of as the points where each operation has an effect; we refer to these as
\emph{linearisation points}. 

A concurrent history might not be complete, i.e.~it might have some pending
executions that have been called but have not returned.  An \emph{extension}
of a history~$h$ is formed by adding zero or more $\return$ events
corresponding to pending executions.  We write $complete(h)$ for the
subsequence of~$h$ formed by removing all $\call$ events corresponding to
pending executions.

We say that a (not necessarily complete) concurrent history~$h$ is
\emph{linearisable} with respect to specification object~$Spec$ if there is an
extension~$h'$ of~$h$ such that $complete(h')$ is linearisable with respect
to~$Spec$.  Informally, the $\return$ events that are in~$h'$ but not~$h$ are
for operation executions that have had an effect, but not returned in~$h$; the
$\call$ events removed in $complete(h')$ are for operation executions that
have not yet had an effect.

We say that a concurrent datatype is linearisable with respect to~$Spec$ if
each of its histories is linearisable with respect to~$Spec$.

%%%%%%%%%%%%%%%%%%%%%%%%%%%%%%%%%%%%%%%%%%%%%%%%%%%%%%%%%%%%

\subsection{Synchronisation linearisation}
\label{sec:sync-lin}

We now adapt the definition of linearisation to synchronisations.  With
standard linearisation, operations appear to take place in a one-at-a-time
order, each between the time at which the operation is invoked and when it
returns.  With synchronisation linearisation, synchronisations appear to take
place in a one-at-a-time order, with each synchronisation between the time
that each of the operations is invoked and when it returns.  In each case,
the order is one that satisfies the requirements captured by the specification
object.  

For the moment, we consider only binary heterogeneous synchronisations; we
generalise in the next section.  We consider a synchronisation object~$Sync$
with two operations, $\op_1$ and~$\op_2$, as described earlier.  A concurrent
history of~$Sync$ contains $\call$ and $\return$ events, as in the previous
subsection, corresponding to the operations~$\op_1$ and~$\op_2$.

For example, the following is a complete history of the synchronous channel
from earlier, and is illustrated in Figure~\ref{fig:sync-timeline}:
\begin{eqnarray*}
h & = & 
\seq{\begin{align}
  \call.\sm{send}^1(8),\; \call.\sm{send}^2(8),\; \call.\sm{receive}^3(()),\;
  \return.\sm{receive}^3\::8,\; \\
  \call.\sm{receive}^4(()),\; \return.\sm{send}^1\::(),\;
  \call.\sm{send}^5(9),\; \return.\sm{receive}^4\::9,\; \\
  \call.\sm{receive}^6(()),\; \return.\sm{send}^2\::(), \;
  \return.\sm{send}^5\::(),\; \return.\sm{receive}^6\::8 } .
  \end{align}
\end{eqnarray*}

%%%%%

\begin{figure}
\unScalaMid
\begin{center}
\begin{tikzpicture}[xscale = 0.9]
\draw[|-|] (0,0) -- node[above] {$\sm{send}^1(8)\::()$} (4,0);
\draw[|-|] (4.5,0) -- node[above] {$\sm{send}^5(9)\::()$} (8.5,0);
\draw[|-|] (0.5,-1) -- node[above] {$\sm{send}^2(8)\::()$} (8,-1);
\draw[|-|] (1,-2) -- node[above] {$\sm{receive}^3(())\::8$} (3,-2);
\draw[|-|] (3.5,-2) -- node[above] {$\sm{receive}^4(())\::9$} (6.5,-2);
\draw[|-|] (7,-2) -- node[above] {$\sm{receive}^6(())\::8$} (9,-2);
\draw (1.9,0) \X; \draw (1.9,-2) \X; 
\draw[dashed] (1.9,0) -- (1.9,-2); % sync 1 and 3
\draw (5.6,0) \X; \draw (5.6,-2) \X; 
\draw[dashed] (5.6,0) -- (5.6,-2); % sync 4 and 5
\draw (7.5,-1) \X; \draw (7.5,-2) \X; 
\draw[dashed] (7.5,-1) -- (7.5,-2); % sync 2 and 6 
\end{tikzpicture}
\end{center}
\caption{Timeline representing the synchronisation example.}
\label{fig:sync-timeline}
\scalaMid
\end{figure}


A history of a synchronisation specification object $Spec$ is a sequence of
events of the form $\sync^{i_1, i_2}(x_1, x_2)\:: (y_1, y_2)$, representing an
execution of |sync| with parameters $(x_1, x_2)$ and result $(y_1,y_2)$.  The
event's  identity is~$(i_1,i_2)$: each of~$i_1$ and~$i_2$ must
appear at most once in the history.  Informally, an event $\sync^{i_1,
  i_2}(x_1, x_2)\:: (y_1, y_2)$ corresponds to a synchronisation between
executions $\op_1^{i_1}(x_1)\::y_1$ and $\op_2^{i_2}(x_2)\::y_2$ in a history
of the corresponding synchronisation object.

A history is \emph{legal} if is consistent with the definition of the
specification object.
%
For example, the following is a legal history of |SyncChanSpec|.
\begin{eqnarray*}
h_s & = & 
\seq{
 \sm{sync}^{1,3}(8,()) \:: ((), 8), \;
 \sm{sync}^{5,4}(9,()) \:: ((), 9), \;
 \sm{sync}^{2,6}(8,()) \:: ((), 8) } .
\end{eqnarray*}
The history is illustrated by the ``$\cross$''s in
Figure~\ref{fig:sync-timeline}: each event corresponds to the synchronisation
of two operations, so is depicted by two ``$\cross$''s on the corresponding
operations, linked by a dashed vertical line.  
%
This particular synchronisation specification object is stateless, so in fact
any permutation of the history~$h_s$ would also be legal (but not all such
permutations will be compatible with the history of the synchronisation
object); but the same will not be true in general of a specification object
with state.
%
The points at which the events of~$h_s$ are inserted into~$h$ can be thought
of as the points where each synchronisation takes place; we refer to these as
\emph{synchronisation points}. 

%
\begin{definition}
Let $h$ be a complete history of the synchronisation object~$Sync$.  We say
that a legal history~$h_s$ of~$Spec$ \emph{corresponds} to~$h$ if their
events agree; more precisely:
\begin{itemize}
\item For each  $\sync^{i_1, i_2}(x_1, x_2)\:: (y_1, y_2)$ in~$h_s$,\,
  $h$ contains events $\call.\op_1^{i_1}(x_1)$,\,
  $\return.\op_1^{i_1}\::y_1$\,, $\call.\op_2^{i_2}(x_2)$, and
  $\return.\op_2^{i_2}\::y_2$.

\item For each $\call.\op_1^{i_1}(x_1)$ and $\return.\op_1^{i_1}\::y_1$
  in~$h$,\, $h_s$ contains an event $\sync^{i_1, i_2}(x_1, x_2)\:: (y_1, y_2)$
  for some~$i_2$, $x_2$, and~$y_2$.

\item For each $\call.\op_2^{i_2}(x_2)$ and $\return.\op_2^{i_2}\::y_2$
  in~$h$,\, $h_s$ contains an event $\sync^{i_1, i_2}(x_1, x_2)\:: (y_1, y_2)$
  for some~$i_1$, $x_1$, and~$y_1$.
\end{itemize}
%% %
%% \begin{itemize}
%% \item For each |sync| event with identity~$(i_1,i_2)$ in~$h_s$,\, $h$ contains
%%   an execution of~$\op_1$ with identity~$i_1$ and an execution of~$\op_2$ with
%%   identity~$i_2$;

%% \item For each execution of~$\op_1$ with identity~$i_1$ in~$h$,\, $h_s$
%%   contains a |sync| event with identity~$(i_1,i_2)$ for some~$i_2$;

%% \item For each execution of~$\op_2$ with identity~$i_2$ in~$h$,\, $h_s$
%%   contains a |sync| event with identity~$(i_1,i_2)$ for some~$i_1$.
%% \end{itemize}
\end{definition}

\begin{definition}
Given a complete history $h$ of~$Sync$ and a corresponding legal history $h_s$
of~$Spec$, we say that $h_s$ is a \emph{synchronisation linearisation} of~$h$
if there is some way of interleaving $h$ and~$h_s$ such that each event
$\sync^{i_1, i_2}(x_1, x_2)\:: (y_1, y_2)$ occurs between
$\call.\op_1^{i_1}(x_1)$ and $\return.\op_1^{i_1}\::y_1$, and between
$\call.\op_2^{i_2}(x_2)$ and $\return.\op_2^{i_2}\::y_2$.
\end{definition}
%
%% We say that $h$ is \emph{synchronisation-linearisable} with respect to~$Spec$
%% in this case.  
In~the running example, $h_s$ is a synchronisation
linearisation of~$h$, as shown by the interleaving in
Figure~\ref{fig:sync-timeline}.

\begin{definition}
\label{def:sync-lin-objects}
Given a (not necessarily complete) concurrent history~$h$ and a corresponding
legal history~$h_s$ of $Spec$, we say that $h_s$ is a \emph{synchronisation
  linearisation} of~$h$ if there is an extension~$h'$ of~$h$ such that $h_s$
is a synchronisation linearisation of $complete(h')$.
%
We say that $h$ is synchronisation-linearisable with respect to~$Spec$ in this
case.  We say that a synchronisation object is synchronisation-linearisable
with respect to~$Spec$ if each of its histories is
synchronisation-linearisable with respect to~$Spec$.
\end{definition}

Informally, the $\return$ events that are
in~$h'$ but not~$h$ are for operation executions that have synchronised, but
not returned in~$h$; the $\call$ events removed in $complete(h')$ are for
operation executions that have not yet synchronised.


