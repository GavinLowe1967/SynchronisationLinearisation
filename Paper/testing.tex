\section{Linearisability testing}
\label{sec:lin-testing}

In the following two sections, we describe techniques for testing whether the
implementation of a synchronisation object is synchronisation linearisable
with respect to a synchronisation specification object.
%
The techniques are influenced by the techniques for testing (standard)
linearisation~\cite{wing-gong,gavin:lin-testing}, so we begin by sketching
those techniques.

The idea of linearisability testing is as follows.  We run several
\emph{worker threads}, performing operations (typically chosen randomly) upon
the concurrent datatype that we are testing, and logging the calls and
returns.  More precisely, a thread that performs a particular
operation~$\sm{op}^i(x)$: (1) writes $\call.\sm{op}^i(x)$ into the log;
(2)~performs $\sm{op}(x)$ on the synchonisation object, obtaining result~$y$,
say; (3)~writes $\return.\sm{op}^i \:: y$ into the log.  Further, the logging
associates each operation execution with an execution $\sm{op}(x)$ of the
corresponding operation on the specification object.

Once all worker threads have finished, we can use an algorithm to test whether
the history is linearisable with respect to the specification object.  The
algorithm searches for an order to linearise the executions, consistent with
the log, and such that the order represents a legal history of the
corresponding executions on the specification object.
See~\cite{wing-gong,gavin:lin-testing} for details of several algorithms.
All these algorithms assume that the specification object is deterministic.

This process can be repeated many times, so as to generate and analyse many
histories.  Our experience is that the technique works well.  It seems
effective at finding bugs, where they exist, typically within a few seconds;
for example, we used it to find an error in the concurrent priority queue
of~\cite{faulty-pri-queue}, which we believe had not previously been
documented.  Further, the technique is easy to use: we have taught it to
undergraduate students, who have used it effectively.

This testing concentrates upon the safety property of
linearisation, rather than liveness properties such as deadlock-freedom.
However, if the concurrent object can deadlock, it is likely that the testing
will discover this.  Related to this point, it is the responsibility of the
tester to define the worker threads in a way that all executions will
eventually return, so the threads terminate.  For example, consider a partial
stack where a |pop| operation blocks while the stack is empty; here, the
tester would need to ensure that threads collectively perform at least as many
|push|es as |pop|s, to ensure that each |pop| does eventually return.

Note also that there is potentially a delay between a worker thread writing the
$\call$ event into the log and actually calling the operation; and likewise
there is potentially a delay between the operation returning and the thread
writing the $\return$ event into the log.  However, these delays do not
generate false errors: if a history without such delays is linearisable, then
so is a corresponding history with delays.  We believe that it is essential
that the technique does not give false errors: an error reported by testing
should represent a real error; testing of a correct implementation should be
able to run unsupervised, maybe for a long time.  Further, our experience is
that the delays do not prevent the detection of bugs when they exist (although
might require performing the test more times).  This means that a failure to
find any bugs, after a large number of tests, can give us good confidence in
the correctness of the concurrent datatype.

%%%%%%%%%%%%%%%%%%%%%%%%%%%%%%%%%%%%%%%%%%%%%%%%%%%%%%%

\section{Adapting the linearisability framework}
\label{sec:testing-hacking}

In this section we investigate how to use the existing linearisation testing
framework for testing synchronisation linearisation, using the ideas of
Section~\ref{sec:relating}.  We require the synchronisation specification
object to be deterministic, reflecting the fact that the linearisation
testing framework requires its specification object to be deterministic.  

This is not a use for which the framework was intended, so we consider it a
hack.  However, it has the advantage of not requiring the implementation of
any new algorithms.  (We do not consider progressibility in this section.)

%% Recall, from the introduction of Section~\ref{sec:relating}, that a
%% straightforward approach won't work.  Instead 

We adapt the idea of two-step linearisation from Section~\ref{sec:relating}.
We start by considering the case of binary heterogeneous synchronisation.  We
aim to obtain a log history that can be tested for (standard) linearisation
against |TwoStepLinSpec|.

As with standard linearisability testing, we run several worker threads,
calling operations on the synchronisation object, and logging the calls and
returns.
%
\begin{itemize}
\item A thread~$t_1$ that performs the concrete operation~$\op_1(x_1)$:
  (1)~writes $\call.\sm{op}_1^{i_1}(x_1)$ into the log, associating it with a
  corresponding execution $\op_1(t_1, x_1)$ on the specification object;
  (2)~performs $\op_1(x_1)$ on the synchonisation object, obtaining
  result~$y_1$, say; (3)~writes $\return.\op_1^{i_1} \:: ()$ into the log;
  (4)~writes $\call.\overline{\op}_1^{i_1}()$ into the log, associating it
  with a corresponding execution $\overline{\op}_1(t)$ on the specification
  object; (5)~writes $\return.\overline{\op}_1^{i_1} \:: y_1$ into the log.

\item A thread~$t_2$ that performs operation~|op|\s2, acts as for standard
  linearisability testing.  It: (1)~writes $\call.\sm{op}_2^{i_2}(x_2)$ into
  the log, associating it with a corresponding execution $\op_2(x_2)$ on the
  specification object; (2)~performs $\op_2(x_2)$ on the synchonisation
  object, obtaining result~$y_2$, say; (3)~writes $\return.\op_2^{i_2} \::
  y_2$ into the log
\end{itemize}
%
The top half of Figure~\ref{fig:twostep-timeline} illustrates a possible run,
containing a single synchronisation, together with the log history.

\begin{figure}
\begin{center}
\def\y{-1.4} % op_2 y-coord
\def\ySLin{-2.5} % sync-lin y-ccord
\def\yLin{-3.6} % linearisation y-coord
\begin{tikzpicture}%[yscale = 1.4]
\bulletAt(-0.6,0){$\call.\op_1(x_1)$};
\draw[|-|] (0.0,0) -- node[above] {$\op_1(x_1)\::y_1$} (2.5,0);
\bulletAt(3.1,0){$\return.\op_1$};
\bulletAt(4.7,0){$\call.\overline{\op}_1$};
\bulletAt(6.5,0){$\return.\overline{\op}_1\::y_1$};
%
\bulletAt(-0.3,\y){$\call.\op_2(x_2)$};
\draw[|-|] (0.4,\y) -- node[above]{$\op_2(x_2)\::y_2$} (2.7,\y);
\bulletAt(3.5,\y){$\return.\op_2\::y_2$};
%
\draw (-2,\ySLin) node{$h_s$};
\crossAt(1.4,\ySLin){$\sync(x_1,x_2)\::(y_1,y_2)$};
%
\draw (-2,\yLin) node{$h_{2s}$:};
\draw(1.2,\yLin) \X; 
\draw(0.5,\yLin-0.4) node{\footnotesize $\op_1(t_1,x_1)$};
\draw(1.6,\yLin) \X;
\draw(2.3,\yLin-0.4) node{\footnotesize $\op_2(x_2)\::y_2$};
%% \crossAt(0.5,\yLin){$\op_1(t_1,x_1)$};
%% \crossAt(2.3,\yLin){$\op_2(x_2)\::y_2$};
\crossAt(5.5,\yLin){$\overline{\op}_1(t_1)\::y_1$};
\end{tikzpicture}
\end{center}
\caption{Illustration of two-step linearisation testing.  The operation
  executions are represented by the horizontal lines with labels above
  (denoted ``$h$'' in Proposition~\ref{prop:twostep-testing}).  The log
  entries are represented by the bullets with labels below (denoted ``$h_l$''
  in Proposition~\ref{prop:twostep-testing}).  Linearisation points are
  represented by crosses with labels below: the penultimate row, labelled
  ``$h_s$'', is a synchronisation linearisation; the bottom row, labelled
  ``$h_{2s}$'', is a linearisation of the two-step synchronisation object.
  Execution identifiers and null arguments and returns are omitted, for
  clarity.}
\label{fig:twostep-timeline}
\end{figure}

%%%%%

Once all threads have finished, we test whether the log history is
linearisable (i.e.~standard linearisation) with respect to |TwoStepLinSpec|
from Section~\ref{sec:relating}.  Figure~\ref{fig:twostep-timeline} gives an
example linearisation, denoted~$h_{2s}$.

Note that we have three related concepts here: (1)~synchronisation
linearisation of the concrete history of operation executions with respect to
|SyncSpec|; (2)~two-step linearisation of the concrete history with respect
to~|TwoStepLinSpec|; and (3)~linearisation of the log history with respect
to~|TwoStepLinSpec|.  Proposition~\ref{prop:two-step-lin} shows that the first
two of these are equivalent.  We need to show that these imply~(3), so the
technique does not give false errors.  (The converse might not hold, because
of delays in writing to the log.)

Since the linearisation algorithm receives  a log history, rather
than a concrete history, we need to describe the relationship.  
%% The following definition captures that a log history~$h_l$ might arise from a
%% concrete history~$h$, using the logging strategy described above.
%
\begin{definition}
Let $h$ be a complete history of a binary heterogeneous synchronisation
object, and let $h_l$ be a log history for the same object.  We say that the
two histories \emph{correspond} if there is some way of interleaving them such
that
%
\begin{itemize}
\item Each $\call.\op_1^{i_1}(x_1)$, from $h_l$, precedes the call and return
  of~$\op_1^{i_1}(x_1)\::y_1$ from~$h$, which precede $\return.\op_1^{i_1} \::
  ()$, $\call.\overline{\op}_1^{i_1}()$ and $\return.\overline{\op}_1^{i_1}
  \:: y_1$, from~$h_l$, in that order.

\item Each $\call.\op_2^{i_2}(x_2)$, from $h_l$, precedes the call and return
  of~$\op_2^{i_2}(x_2)\::y_2$ from~$h$, which precede $\return.\op_2^{i_2}
  \:: y_2$, from~$h_l$.
\end{itemize}
\end{definition}

\begin{prop}
\label{prop:twostep-testing}
Let $h$ be a complete history of a binary heterogeneous synchronisation
object, and let $h_l$ be a corresponding log history for the same object.  Let
|SyncSpec| be a synchronisation specification object, and |TwoStepSyncSpec|
the corresponding two-step synchronisation specification object, constructed
as in Section~\ref{sec:twoStepLinSpec}.  Suppose $h$ is
synchronisation-linearisable with respect to |SyncSpec|.  Then $h_l$ is
linearisable with respect to |TwoStepSyncSpec|.
\end{prop}
%
\begin{proof}
Since $h$ is synchronisation-linearisable, there is a legal history~$h_s$ of
|SyncSpec| such that $h_s$ is a synchronisation linearisation of $h$.
Consider the interleaving of $h_s$ and~$h$, that demonstrates this, and
interleave $h_l$ with it, consistent with the interleaving of~$h$ and~$h_l$
that demonstrates that they correspond.  Figure~\ref{fig:twostep-timeline}
illustrates such an interleaving.

We build a history~$h_{2s}$ of~|TwoStepSyncSpec|, and interleave it with~$h_l$
as follows.  In the interleaving of the previous paragraph, replace each event
$\sync^{i_1,i_2}(x_1,x_2)\::(y_1,y_2)$ (from~$h_s$) by immediately consecutive
events~$\op_1^{i_1}(x_1)\::()$ and $\op_2^{i_2}(x_2)\::y_2$, and add
$\overline{\op}_1^{i_1}()\::y_1$ between $\call.\overline{\op}_1^{i_1}()$ and
$\return.\overline{\op}_1^{i_1} \:: y_1$ (from~$h_l$).  Again,
Figure~\ref{fig:twostep-timeline} illustrates such an interleaving.  This is a
legal history of~|TwoStepSyncSpec|, by
Lemma~\ref{lem:TwoStepLinSpec-histories}.  Further, each event of~$h_{2s}$ is
between the corresponding $\call$ and $\return$ events of~$h_l$, by
construction.  Hence $h_{2s}$ is a linearisation of~$h_l$.
\end{proof}

%%%%%%%%%%%%%%%%%%%%%%%%%%%%%%%%%%%%%%%%%%%%%%%%%%%%%%%

%% \subsection{OLD VERSION}

%% Note there might be delays involved in writing to the log, so that the log
%% history does not correspond precisely to the history of operation calls and
%% returns.  We show that the approximation serves our purposes.

%% Consider a history~$h$ of the synchronisation object, and suppose, for the
%% moment, there are no delays in logging, i.e.:\ (1)~the $\call.\op_1$ and
%% $\call.\op_2$ events happen immediately before the actual calls; (2)~the
%% $\return.\op_2$ events happen immediately after the return of~$\op_2$; and
%% (3)~the $\call.\overline\op_1$ and $\return.\overline\op_1$ events happen
%% immediately after the return of $\op_1$ --- in each case with no intervening
%% events.  Then the log history is linearisable with respect to |TwoStepLinSpec|
%% if and only if the history~$h$ is two-step linearisable with respect to
%% |TwoStepLinSpec|.  But Proposition~\ref{prop:two-step-lin} then shows that
%% this holds if and only if $h$ is synchronisation linearisable.  In particular,
%% this shows that errors are detected providing the logging is fast enough.

%% We now show that delays in logging do not introduce false errors.  Consider a
%% history~$h$ that is synchronisation linearisable, and consider the
%% corresponding log history~$h_l$.  Now build a history~$h'$ so that each
%% operation is extended so that: (1)~each call of~$\op_1$ or~$\op_2$ is
%% immediately before the $\call.\op_1$ or $\call.\op_2$ event in~$h_l$; (2)~each
%% return of~$\op_1$ is immediately after the $\return.\overline\op_1$ event
%% in~$h_l$; and (3)~each return of~$\op_2$ is immediately after the
%% $\return.\op_2$ event in~$h_l$.  This construction is illustrated below for
%% the two types of operation.
%% %
%% \begin{center}
%% \begin{tikzpicture}[xscale = 1.0, yscale = 1.0]
%% \draw (-2.0,0) node{$h$:};
%% \draw (-2.0,-1) node{$h_l$:};
%% \draw (-2.0,-2.3) node{$h'$:};
%% % op_1^1
%% \draw[|-|] (0,0) -- node[above] {$\sm{op}_1(x_1)\::y_1$} (1,0);
%% \bulletAt(-0.2,-0.8){$\call.\sm{op}_1$}
%% \bulletAt(1.2,-0.8){$\return.\sm{op}_1$}
%% \bulletAt(2.7,-0.8){$\call.\overline{\op}_1$}
%% \bulletAt(4.2,-0.8){$\return.\overline{\op}_1$}
%% \draw[|-|] (-0.4,-2.3) -- node[above] {$\sm{op}_1(x_1)\::y_1$} (4.4,-2.3);
%% %%%%%
%% \draw[|-|] (7.0,0) -- node[above] {$\sm{op}_2(x_2)\::y_2$} (8,0);
%% \bulletAt(6.8,-0.8){$\call.\sm{op}_2$}
%% \bulletAt(8.2,-0.8){$\return.\op_2$}
%% \draw[|-|] (6.6,-2.3) -- node[above] {$\sm{op}_2(x_2)\::y_2$} (8.4,-2.3);
%% \end{tikzpicture}
%% \end{center}
%% %
%% Now $h$ is synchronisation linearisable; and hence $h'$ is also (since each
%% operation in $h'$ is an extension of the corresponding operation in~$h$).  But
%% then Proposition~\ref{prop:two-step-lin} implies that $h'$ is two-step
%% linearisable with respect to |TwoStepLinSpec|.  Hence $h_l$ is linearisable
%% with respect to |TwoStepLinSpec|, by construction.

This approach generalises to non-binary synchronisations, homogeneous
synchronisations, and stateful specification objects as in
Section~\ref{ssec:relating-variations}.


As with standard linearisation, the tester needs to define the worker threads
so that all executions will eventually return, i.e.~so that each will be able
to synchronise.  For a binary heterogeneous synchronisation with no
precondition, we can achieve this by half the threads calling one operation,
and the other half calling the other operation (with the same number of calls
by each).  For a binary homogeneous synchronisation, this approach might not
work if every worker does more than one operation: one worker might end up
with two operations to perform, when all others have terminated; instead, we
arrange for an even number of workers to each perform a single operation.


%%%%%%%%%%%%%%%%%%%%%%%%%%%%%%%%%%%%%%%%%%%%%%%%%%%%%%%

\paragraph{Variable-arity synchronisations}

It turns out that it is not, in general, possible to capture variable-arity
synchronisations using this technique, in particular where the arity of a
synchronisation depends upon the relative timing of executions, as opposed to
the state of the specification object.  This is a result of two things: that
the logging of operations, in particular the $\overline{\op}_1$, can be
arbitrarily delayed; and that it can be nondeterministic whether or not two
executions synchronise, which is at odds with the fact that each operation on
the specification object needs to be deterministic.

To illustrate this point, consider a timeout channel.  Without loss of
generality, let the |send| operation correspond to $\op_1$, and the
|receive| operation correspond to~$\op_2$.

The top-half of Figure~\ref{fig:two-step-timeout-channel} gives a
timeline illustrating a successful |send(3)| and |receive|.  This
corresponds to the history
\[
\seq{ \send(3)\::(),\; \receive()\::\sm{Some}(3),\; 
  \overline{\send}()\::\sm{true} }
\]
of the specification object.

%%%%%%%%%%

\begin{figure}
\begin{center}
\def\yLin{-2.5} % y-coord for linearisation
\begin{tikzpicture}[xscale = 0.9]
\bulletAt(-0.4,0){$\call.\send(3)$};
\draw[|-|] (0,0) -- node[above] {$\send(3)\::()$} (1.6,0);
\bulletAt(2.0,0){$\return.\send$};
%
\bulletAt(3.9,0){$\call.\overline{\send}$};
\bulletAt(6.2,0){$\return.\overline{\send}\::\sm{true}$};
%
\bulletAt(-0.5,-1.5){$\call.\receive\qquad$};
\draw[|-|] (0.3,-1.5) -- 
  node[above] {$\receive()\::\sm{Some}(3)$} (1.9,-1.5);
\bulletAt(2.8,-1.5){\ $\qquad\qquad\return.\receive\::\sm{Some(3)}$};
%
\draw(0.7,\yLin) \X; 
\draw(-0.1,\yLin-0.4) node{\footnotesize $\send(3)\::()$};
\draw(1.1,\yLin) \X;
\draw(2.6,\yLin-0.4) node{\footnotesize $\receive()\::\sm{Some}(3)$};
%% \crossAt(0.7,-3){$\send(3)\::()$}
%% \crossAt(1.5,-3){$\receive()\::\sm{Some}(3)$}
\crossAt(5.5,\yLin){$\overline{\send}\::\sm{true}$}
%
\draw (-2.2,-3.6) -- ++(12.4,0); 
\end{tikzpicture}

%%%%%%
%\hfil\hfil
\medskip

\begin{tikzpicture}[xscale = 0.9]
\bulletAt(-0.4,0){$\call.\send(3)$};
\draw[|-|] (0,0) -- node[above] {$\send(3)\::()$} (1.6,0);
\bulletAt(2.0,0){$\return.\send$};
%
\bulletAt(7.1,0){$\call.\overline{\send}$};
\bulletAt(9.4,0){$\return.\overline{\send}\::\sm{false}$};
%
\bulletAt(2.5,-1.5){$\call.\receive\qquad$};
\draw[|-|] (3.3,-1.5) -- 
  node[above] {$\receive()\::\sm{None}$} (4.9,-1.5);
\bulletAt(5.8,-1.5){\ $\qquad\qquad\return.\receive\::\sm{None}$};
%
\crossAt(0.8,\yLin){$\send(3)\::()$};
\crossAt(3.9,\yLin){$\receive()\::\sm{Some}(3)$};
\crossAt(8.2,\yLin){$\overline{\send}\::\sm{true}$}
%% \draw[|-|] (0,-3.5) -- node[above] {$\send(3)\::()$} (1.6,-3.5);
%% \crossAt(0.7,-3.5){$\send(3)\::()$}
%% %
%% \draw[|-|] (2.2,-5) -- 
%%   node[above] {$\receive()\::\sm{None}$} (3.8,-5);
%% \crossAt(3.4,-5){$\receive()\::\sm{Some}(3)$}
%% %
%% \draw[|-|] (4.0,-3.5) -- node[above] 
%%   {$\overline{\send}(3)\::\sm{false}$} (5.6,-3.5);
%% \crossAt(4.8,-3.5){$\overline{\send}(3)\::\sm{true}$}
\end{tikzpicture}
\end{center}
\caption{Figure showing why two-step linearisation cannot be used for a
  timeout channel.  Conventions are as in Figure~\ref{fig:twostep-timeline}.}
\label{fig:two-step-timeout-channel}
\end{figure}

%%%%%%%%%%

The bottom-half side of Figure~\ref{fig:two-step-timeout-channel} gives a
timeline illustrating an unsuccessful |send(3)| and |receive|, and where the
logging of $\overline{\send}$ is delayed.  None of the executions overlap, so
they must necessarily be linearised in the same order as in the previous
history.  The specification object is deterministic, so the operations must
return the same results as in the previous history.  But, in the cases of
$\receive$ and $\overline{\send}$, those returned values, |Some(3)| and
|true|, do not agree with the corresponding values in the log history, |None|
and |false|.  Hence the history would be flagged as an error, despite being
valid.

The difference between this situation and the discussion in
Section~\ref{ssec:relating-variations} is that the logging of
operations, in particular the $\overline{\send}$, can be arbitrarily delayed.
However, in the earlier section we allowed the $\overline{\send}$ anywhere
within the corresponding concrete operation.  This means that a history like
in the bottom-half of Figure~\ref{fig:two-step-timeout-channel} could be
linearised by the history
\[
\seq{ \send(3)\::(),\; \overline{\send}(3)\::\sm{false},\;
  \receive()\::\sm{None} }
\]
of the two-step specification object.  Here the operations take place in a
different order than for Figure~\ref{fig:two-step-timeout-channel}, and so
this is consistent with a deterministic specification object.

A similar problem arises with a timeout exchanger.

%\framebox{Explain how it can be done.}

We have investigated an alternative approach, which involves worker threads
adapting their logging behaviour based on the outcome of their operations.
For the timeout channel:
\begin{itemize}
\item A thread that sends a value~$x$: (1)~writes $\call.\send^{i_1}(x)$ into
  the log; (2)~performs $\send(x)$ on the channel; (3)~writes
  $\return.\send^{i_1} \:: ()$ into the log; (4)~if the send is successful,
  associates the log entries with an operation $\send(x)$ on the specification
  object, and otherwise associates them with an operation $\sm{sendFail}(x)$;
  (5)~if the send is successful, writes $\call.\overline{\send}^{i_1}()$ and
  $\return.\overline{\send}_1^{i_1} \:: ()$ into the log, associating them
  with an operation $\overline{\send}()$ on the specification object (and
  otherwise does nothing).

%%     (6)~writes $\return.\overline{\send}_1^{i_1} \:: ()$ into the log

%% then:
%%   \begin{itemize}
%%   \item if the send is successful: (4) associates the log entries with an
%%     operation $\send(x)$ on the specification object; (5)~writes
%%     $\call.\overline{\send}^{i_1}()$ into the log, associating it with a
%%     corresponding execution $\overline{\send}()$ on the specification object;
%%     (6)~writes $\return.\overline{\send}_1^{i_1} \:: ()$ into the log.

%%   \item if the send is unsuccessful: (4)~associates the log entries with an
%%     operation $\sm{sendFail(x)}$ on the specification object (and does not
%%     perform the second step).
%%   \end{itemize}

\item A thread that performs a receive: (1)~writes $\call.\receive^{i_2}$ into
  the log; (2)~performs $\receive$ on the channel, receiving result~$r$, say;
  (3)~writes $\return.\receive^{i_1} \:: r$ into the log; (4)~if the receive
  was successful, associates the log entries with an operation |receive| on
  the specification, and otherwise associates them with an
  operation $\sm{receiveFail}$.
\end{itemize}



\begin{window}[0,r,{
\begin{minipage}{48.5mm}
\begin{tikzpicture}[>= angle 60]
\draw (0,0) node[draw] (zero) {$\sm{Zero}$};
\draw[<-] (zero) -- ++ (-1.5,0);
\loopRight(zero){$
%% \draw[->] (zero) .. controls ++(1.5,0.4) and (1.5,-0.4) .. node[right] {$
  \begin{align} 
  \sm{sendFail}, \\ \overline{\send}, \\ \sm{receiveFail}
  \end{align}$}% (zero);
%
\draw (0,-2) node[draw] (one) {$\sm{One}(\sm{x})$};
\draw[->] (zero) .. controls ++(0.3,-1) .. node[right] {$\sm{send(x)}$} (one); 
\draw[->] (one) .. controls ++(-0.3,1) .. 
  node[left] {$\begin{align}\sm{receive}\::\\ \quad\sm{Some(x)}\end{align}$} (zero);
\end{tikzpicture}
\end{minipage}
},] 
The specification object then encodes the automaton to the right.
Thus a successful synchronisation is linearised by the sequence $\send(\sm x)$,
$\receive\::\sm{Some(x)}$, $\overline{\send}$, with the former two events
consecutive, as for other binary heterogeneous synchronisations.  Unsuccessful
sends and receives are each linearised by a single event.
\end{window}

%%%%%%%%%%%%%%%%%%%%%%%%%%%%%%%%%%%%%%%%%%%%%%%%%%%%%%%

%% The specification object then encodes the automaton below. 
%% %% \begin{window}[0,r,{
%% %% \begin{minipage}{73mm}
%% \begin{center}
%% \begin{tikzpicture}[>= angle 60]
%% \draw (0,0) node[draw] (zero) {$\sm{Zero}$};
%% \draw[<-] (zero) -- ++ (-1.5,0);
%% \loopAbove(zero){$
%% %% \draw[->] (zero) .. controls ++(1.5,0.4) and (1.5,-0.4) .. node[right] {$
%%   \begin{array}{c} 
%%   \sm{sendFail}, \overline{\send}, \\ \sm{receiveFail}
%%   \end{array}$}% (zero);
%% %
%% \draw (4,0) node[draw] (one) {$\sm{One}(\sm{x})$};
%% \draw[->] (zero) .. controls ++(2,0.3) .. node[above] {$\sm{send(x)}$} (one); 
%% \draw[->] (one) .. controls ++(-2,-0.3) .. 
%%   node[below] {$\sm{receive}\::\sm{Some(x)}$} (zero);
%% \end{tikzpicture}
%% \end{center}
%% %% \end{minipage}
%% %% },] 
%% Thus a successful synchronisation is linearised by the sequence $\send(\sm x)$,
%% $\receive\::\sm{Some(x)}$, $\overline{\send}$, with the former two events
%% consecutive, as for other binary heterogeneous synchronisations.  Unsuccessful
%% sends and receives are each linearised by a single event.

A similar technique can be used for a timeout exchanger.  We consider this
approach convoluted, and we do not advocate it.  We include it only for
completeness.

