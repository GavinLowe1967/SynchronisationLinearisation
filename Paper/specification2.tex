
\subsection{Variations}
\label{ssec:spec-variations}

Above we considered heterogeneous binary synchronisations, i.e.~two
invocations of different operations, with a single mode of synchronisation.

It is straightforward to generalise to synchronisations between an arbitrary
number of invocations, some of which might be invocations of the same
operation.  Consider a $k$-way synchronisation between operations
\begin{scala}
  def op£\s j£(x£\s j£: A£\s j£): B£\s j£   £for $j = 1, \ldots, k,$£
\end{scala}
%
where the $\op_j$ might not be distinct.
The specification object will have a corresponding operation
%
\begin{scala} 
  def sync(x£\s1£: A£\s1£, ..., x£\s k£: A£\s k£): (B£\s1£, ..., B£\s k£)
\end{scala}
%
For example, for the combining barrier |CombiningBarrier(n, f)| of the
Introduction, the corresponding specification object would be
\begin{scala}
class CombiningBarrierSpec{
  def sync(x£\s1£: A, ..., x£\s n£: A) = {
    val result = f(x£\s1£, f(x£\s2£, ...f(x£\s{{\ss n}-1}£, x£\s{\ss n}£)...)); (result,...,result)
  }
}
\end{scala}

A history of the specification object will have corresponding events
$\sm{sync}^{i_1, \ldots, i_k}(x_1, \ldots, x_k)\:: (y_1, \ldots, y_k)$.
%
The definition of synchronisation compatibility is an obvious adaptation of
earlier: in the interleaving of the complete history of the synchronisation
history and the history of the specification object, each $\sm{sync}^{i_1,
  \ldots, i_k}(x_1, \ldots, x_k)\:: (y_1, \ldots, y_k)$ occurs between
$\call.\op_1^{i_j}(x_j)$ and $\return.\op_j^{i_j}\::y_j$ for each $j = 1, \ldots,
k$.  The definition of synchronisation-linearisability follows in the obvious
way. 

It is also straightforward to adapt the definitions to deal with multiple
modes of synchronisation: the specification object has a different operation
for each mode.  For example, recall the |TimeoutChannel| from the
Introduction, where sends and receives may timeout and return without
synchronisation.  The corresponding specification object would be:
%
\begin{scala}
class TimeoutChannelSpec{
  def sync£\s s£(x: A) = false
  def sync£\s r£(u: Unit) = None
  def sync£\s{s,r}£(x: A, u: Unit) = (true, Some(x))
}
\end{scala}
%
The operation $\sm{sync}_s$ corresponds to a send returning without
synchronising; likewise $\sm{sync}_r$ corresponds to a receive returning
without synchronising; and $\sm{sync}_{s,r}$ corresponds to a send and receive
synchronising.  The formal definition of synchronisation linearisation is the
obvious adaptation of the earlier definition.


%%%%%%%%%%%%%%%%%%%%%%%%%%%%%%%%%%%%%%%%%%%%%%%%%%%%%%%

\subsection{Specifying progress}

We now consider a progress condition for synchronisation objects.  

We assume that each pending invocation is scheduled infinitely often, unless
it is blocked (for example, trying to obtain a lock).  Under this assumption,
our progress condition can be stated informally as:
%
\begin{itemize}
\item If a set of pending invocations can synchronise, then some such set
  should eventually synchronise;

\item Once a particular invocation has synchronised, it should eventually
  return.
\end{itemize}
%
Note that there might be several different synchronisations possible.  For
example, consider a synchronous channel, and suppose there are pending calls
to |send(3)|, |send(4)| and |receive|.  Then the |receive| could synchronise
with \emph{either} |send|, nondeterministically; subsequently, the |receive|
should return the appropriate value, and the corresponding |send| should also
return.  In such cases, our progress condition allows \emph{either}
synchronisation to occur.

Our progress condition allows all pending invocations to block if no
synchronisation is possible.  For example, if every pending invocations on a
synchronous channel is a |send|, then clearly none can return.

Note that our progress condition is somewhat different from the condition of
\emph{lock freedom} for concurrent datatypes~\cite{herlihy-shavit}.  That
condition requires that, assuming invocations collectively are scheduled
infinitely often, then eventually some invocation returns.  Lock freedom makes
no assumption about scheduling being fair.  For example, if a particular
invocation holds a lock then it allows the scheduler to never schedule that
invocation; in most cases, this will mean that no invocation returns: an
implementation that uses a lock in a non-trivial way is not lock-free.

By contrast, our assumption, that each unblocked pending invocation is
scheduled infinitely often, reflects that modern schedulers \emph{are} fair,
and do not starve any single invocation.  For example, if an invocation holds
a lock, and is not in a potentially unbounded loop, then it will be scheduled
sufficiently often that it eventually releases the lock.  Thus our progress
condition can be satisfied by an implementation that uses locks.  However, our
assumption does allow invocations to be scheduled in an unfortunate order (as
long as each is scheduled infinitely often).

We make clear what we mean by saying that an invocation can eventually return.
%
\begin{definition}
We say that an infinite execution is \emph{fair} if every  invocation
either returns or performs infinitely many steps.

Consider a state~$st$ of a synchronisation object.  We say that the object can
\emph{eventually return} if for every fair infinite execution from~$st$ that
contains no $\call$ event, there is a $\return$ event.
%% in every state~$st'$ reachable from~$st$ via an execution with no $\call$
%% or $\return$ event, there is an execution from~$st'$ where the next event
%% is a $\return$ event.
\end{definition}
%
Note that the condition means that a return event can happen next on
\emph{every} execution path.


%% \begin{definition}
%% Let $Sync$ be a synchronisation object, and let $h$ be a
%% synchronisation-linearisable history of~$Sync$.  Suppose $h'$ is a proper
%% extension of~$h$ that is synchronisation-linearisable and contains no new
%% $\call$ events.  We say that $Sync$ is \emph{progressive} after~$h$ if it does
%% not block after~$h$: eventually some pending invocation returns.  We say that
%% $Sync$ is \emph{progressive} if it is progressive after each of its
%% synchronisation-linearisable histories.
%% \end{definition}

%% *** That's not quite right.  There might be different histories of the spec
%% object, and only some of them might allow returns; and we want all of the
%% returns to happen.


%% Second attempt.

The following definition describes the circumstances under which it is
acceptable for an object to block, and so does not eventually return.
%
\begin{definition}
Let $Sync$ be a synchronisation object that is synchronisation-linearisable
with respect to specification object $Spec$.  Let $h$ be a history of~$Sync$.
We say that $Sync$ \emph{may block} after~$h$ if there is a valid
history~$h_s$ of~$Spec$, such that:
%
\begin{itemize}
\item $complete(h)$ and~$h_s$ are compatible;

\item There is no proper extension~$h_e$ of~$h$ (adding one or more $\return$
  events) and synchronisation event~$sync$ such that $complete(h_e)$ and~$h_s
  \cat \seq{sync}$ are compatible.
\end{itemize}
\end{definition}

Note that the first condition says that for every synchronisation in~$h_s$,
there are corresponding $\return$ events in~$h$: there is no invocation that
has synchronised but not yet returned.  The second condition says that no more
synchronisations are possible: such a synchronisation would correspond to a
synchronisation event~$sync$.


We give two examples, both for a synchronous channel.
%
\begin{example}
Consider $h = \seq{ \call.\send^1(3), \call.\receive^2(()),
  \return.\receive^2\::3 }$.  There is no history~$h_s$ of~$Spec$ such that
$complete(h) = \seq{\call.\receive^2(()),\linebreak[1] \return.\receive^2\::3
}$ and~$h_s$ are compatible.  (However, the extension $h_e = h \cat \seq{
  \return.\send^1 \:: () }$ is compatible with $h_s =
\seq{\sync^{1,2}(3,())\::((),3) }$).  Informally, the channel may not block
after~$h$ because a synchronisation has occurred, and so there is a pending
return of the send invocation.
\end{example}

\begin{example}
Now consider $h = \seq{\call.\send^1(3), \call.\receive^2(())}$.  We have that
$complete(h) = \seq{}$ is compatible with the history $h_s = \seq{}$ of $Spec$
(and no other). But the extension $h_e = h \cat
\seq{\return.\send^1\::(),\linebreak[1] \return.\receive^2\::3}$ is compatible
with the extension $h_s' = \linebreak[1] \seq{\sync^{1,2}(3,()) \:: ((),3)}$
of~$h_s$.  Hence the channel may not block after~$h$.  Informally, the two
pending invocations can synchronise and then return. 
\end{example}



\begin{definition}
Let $Sync$ be a synchronisation object that is synchronisation-linearisable
with respect to specification object $Spec$.  We say that $Sync$ is
\emph{progressable} if for every history~$h$, if it is not the case that $Sync$
may block after~$h$, then $Sync$ can eventually return from each state reached
after~$h$.
\end{definition}

%  Example with nondeterministic synchronisations?



%% We want to say that all relevant returns are possible, at least in the CSP
%% models.  I don't think we want to make that distinction here.  It won't be
%% true if a return happens before the lock is released.  In the CSP models,
%% we arrange for the end events to be the final events, so after the lock if
%% released.  That also corresponds to what happens at the (virtual-) machine
%% level, if not the high-level language level.  With this placement of end
%% events, the two properties are equivalent. 

