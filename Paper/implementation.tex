\section{Implementation}
\label{sec:implementation}

We have implemented a testing framework (in Scala).  The framework supports
both two-step linearisation and the direct algorithms.  We have used the
framework to implement testers for particular synchronisation
objects\footnote{The implementation is available from \framebox{???}.}.  We
consider the framework to be straightforward to use: most of the boilerplate
code is encapsulated within the framework; defining a tester for a new
synchronisation object normally takes just a few minutes.  Below, we
concentrate our discussion on the part of the framework using the direct
algorithms.

Figure~\ref{fig:ChanTester} gives a stripped-down tester for a synchronous
channel.  (The full version can be used to test several different
implementations with the same interface, and replaces the numeric constants by
parameters that can be set on the command line.)

%%%%%

\begin{figure}
\begin{scala}
object ChanTester extends Tester{
  trait Op   // Representation of operations within the log.
  case class Send(x: Int) extends Op
  case object Receive extends Op

  def worker(c: SyncChan[Int])(me: Int, log: HistoryLog[Op]) = 
    for(i <- 0 until 10)
      if(me%2 == 0) log(me, c.receive(), Receive)
      else{ val x = Random.nextInt(100); log(me, c.send(x), Send(x)) }

  object SyncChanSpec{
    def sync(x: Int, u: Unit) = ((), x)    
  }

  def matching: PartialFunction[(Op,Op), (Any,Any)] = {
    case (Send(x), Receive) => SyncChanSpec.sync(x, ()) // £= ((), x)£.
  }

  /** Do a single test.  Return true if it passes. */
  def doTest(): Boolean = {
    val c = new SyncChan[Int]
    new BinaryStatelessTester[Op](worker(c), 4, matching)()
  }

  def main(args: Array[String]) = {
    var i = 0; while(i < 5000 && doTest()) i += 1
  }
}
\end{scala}
\caption{A simple tester for a synchronous channel. \label{fig:ChanTester}}
\end{figure}

%%%%%%%%%%

The |worker| function defines a worker thread that performs operations on the
channel~|c|.  The function also takes parameters representing the thread's
identity and a log object.  Here, each worker with an even identity performs
10 |receive|s: the call |log(me, c.receive(), Receive)| logs the
call, performs the |receive|, and then logs the return.  Similarly, each
worker with an odd identity performs 10 |send|s of random values.
This definition is designed so that an even number of workers with contiguous
identities will not deadlock. 

|SyncChanSpec| is the synchronisation specification object from earlier.  The
way executions synchronise is captured by the function |matching|.  This is a
partial function whose domain defines which operation executions can
synchronise together, and, in that case, the value each should return: here
|send(x)| and |receive| can synchronise, giving a result as defined by the
synchronisation specification object.  (Alternatively, the call to
|SyncChanSpec.sync| could be in-lined.)

The function |doTest| performs a single test.  This uses a
|BinaryStatelessTester| object from the testing framework, which encapsulates
the search from Section~\ref{sec:binary-heterogeneous}.  Here, the tester runs
4 |worker| threads, and tests the resulting history against |matching|.  If a
non-synchronisation-linearisable history is recorded, it displays this for the
user.  The |main| function runs |doTest| either 5000 times or until an error
is found.  The tester can be adapted to test for synchronisation
progressibility by passing a timeout duration to the |BinaryStatelessTester|.

Other classes of testers are similar.  In the case of a stateful
specification, the |matching| function takes the specification object as a
parameter, and also returns the new value of the specification object.  The
framework directly supports two-step linearisation testing for binary
synchronisations, but for other forms of synchronisation, the programmer has
to define the appropriate automaton.


\begin{figure}
\begin{verbatim}
0:   Call of Exchange(13)
1:   Call of Exchange(70)
1:   Return of 13 from Exchange(70)
2:   Call of Exchange(76)
3:   Call of Exchange(58)
3:   Return of 76 from Exchange(58)
2:   Return of 58 from Exchange(76)
0:   Return of 58 from Exchange(13)
Invocation 0 does not synchronise with any other operation.
\end{verbatim}
\caption{A faulty history for an exchanger, showing a failure of
  synchronisation linearisability.  The left-hand column gives an index for
  each operation execution.}
% scala -cp .:/home/gavin/Scala/SCL  synchronisationTester.ExchangerTester --faulty -p 4
\label{fig:exchanger-error}
\end{figure}

%%%%%

If a tester finds a history that is not synchronisation linearisable, it
displays it.  For example, Figure~\ref{fig:exchanger-error} gives such a
history for an exchanger.  Here executions~2 and~3 have correctly synchronised
and exchanged their values, 76 and~58.  Execution~1 has received execution~0's
value,~13.  However, execution~0 has received execution~3's value, rather than
execution~1's.  This does not, of course, identify what the bug is; but it does
give some indication as to what has gone wrong, namely that execution~1 has been
delayed and so failed to pick up the correct value. 

Similarly, if a tester finds a failure of synchronisation progressibility, it
displays the history.  Figure~\ref{fig:progress-error} gives an example for a
faulty synchronous channel.  Here executions~0 and 1 have successfully
synchronised.  However, executions~2 and~3 have failed to synchronise when
they should have done.

%%%%%

\begin{figure}
\begin{verbatim}
0:   Call of Send(48)
1:   Call of Receive
2:   Call of Send(92)
3:   Call of Receive
1:   Return of 48 from Receive
0:   Return of () from Send(48)
Pending invocations 2 and 3 should have synchronised.
\end{verbatim}
\caption{A faulty history for a synchronous channel, showing a failure of
  progressibility.}
% scala -cp .:/home/gavin/Scala/SCL  synchronisationTester.ChanTester --faulty2 --progressCheck 120 --iters 1 -p 4
\label{fig:progress-error}
\end{figure}
