\section{Model checking for synchronisation linearisation}
\label{sec:modelChecking}

In this section we describe how to analyse a synchronisation object using
model checking, to gain assurance that it satisfies synchronisation
linearisation.  We present our approach within the framework of the process
algebra CSP~\cite{awr:ucs} and its model checker FDR~\cite{fdr3,fdr-manual}.
We assume some familiarity with the syntax of CSP.

In particular, we use checks within the traces model of CSP\null.  This model
represents a process~$P$ by its traces, denoted $traces(P)$, i.e.~the finite
sequences of visible events that~$P$ can perform.  Given processes $P$
and~$Q$, FDR can test whether $traces(P) \subseteq traces(Q)$.  Here $P$ is
typically a model of some system that we want to analyse, and $Q$ is a
specification process that has precisely the traces that correspond to the
desired property.

\framebox{Limitations} of model checking.

We describe how to test for synchronisation linearisation within this
framework.  We start with the case of heterogeneous binary synchronisations;
we describe how to generalise at the end of this section.

We build a CSP model of the synchronisation object.  Such modelling is well
understood, so we don't elaborate in detail.  Typically CSP processes
representing threads perform events to read or write shared variables, acquire
or release locks, etc.  The shared variables, locks, etc., are also
represented by CSP processes.  An example for a synchronous channel can be
found in~\cite{gavin:syncChan}.  

We assume that the model includes the following events:
%
\begin{itemize}
\item \CSPM{call}$.t.op.x$ to represent thread~$t$ calling operation~$op$ with
  parameter~$x$; 

\item \CSPM{return}$.t.op.y$ to represent thread~$t$ returning from
  operation~$op$ with result~$y$.
\end{itemize}
%
We assume that all other events, describing the internal operation of the
synchronisation object, are hidden, i.e.~converted into internal events.

We now describe how to test whether the model satisfies synchronisation
linearisation with respect to a specification object.  We build a process
\CSPM{SyncSpec} corresponding to the specification object.  We assume this
process uses events of the form \CSPM{sync}$.t_1.t_2.x_1.x_2.y_1.y_2$ to
represent a synchronisation between threads~$t_1$ and~$t_2$, calling
$op_1(x_1)$ and~$op_2(x_2)$, and receiving results~$y_1$ and~$y_2$,
respectively.  For example, for the synchronous channel, we would have
%
\begin{cspm}
SyncSpec = sync?t1?t2?x?u!u!x -> SyncSpec
\end{cspm}

If the synchronisation object or specification object has unbounded state, we
have no chance of modelling it using finite-state model checking.  However, we
can often build approximations.  For example, we could approximate (in an
informal sense) the synchronous channel with sequence counter by one where the
sequence counter is stored mod 5.  Then the specification object can
be modelled by
%
\begin{cspm}
SyncSpec = SyncSpec'(1)
SyncSpec'(ctr) = sync?t1?t2?x?u!ctr!(x,ctr) -> SyncSpec'((ctr+1)%5)
\end{cspm}

We then build a \emph{lineariser} process for each thread as follows.
%
\begin{cspm}
Lineariser(t) = 
  call.t.op£\s1£?x£\s1£ -> sync.t?t£\s2£!x£\s1£?x£\s2£?y£\s1£?y£\s2£ -> return.t.op£\s1£.y£\s1£ -> Lineariser(t)
  []
  call.t.op£\s2£?x£\s2£ -> sync?t£\s1£!t?x£\s1£!x£\s2£?y£\s1£?y£\s2£ -> return.t.op£\s2£.y£\s2£ -> Lineariser(t)
alpha(t) = {| call.t, return.t, sync.t.t£\s1£, sync.t£\s1£.t | t£\s1£ <- ThreadID, t£\s1£ != t |} 
\end{cspm}
%
This ensures that between each \CSPM{call} and \CSPM{return} event
of~\CSPM{t}, there is a corresponding \CSPM{sync} event.  

We then combine together the specification process with the linearisers,
synchronising on shared events: this means that each
\CSPM{sync.t}\s1\CSPM{.t}\s2 event will be a three-way synchronisation between
\CSPM{SyncSpec}, \CSPM{Lineariser(t}\s1\CSPM{)} and
\CSPM{Lineariser(t}\s2\CSPM{)}.  
\begin{cspm}
Spec£\s0£ = SyncSpec [| {| sync |} |] (**|| t <- ThreadID @ [alpha(t)] Lineariser(t))
\end{cspm}
Every trace will represent an interleaving
between a possible history of the concurrent object and a legal history of the
specification object.
%
Finally, we hide the \CSPM{sync} events. 
\begin{cspm}
Spec = Spec£\s0£ \ {| sync |}
\end{cspm}
%
Each trace of the resulting process represents a history for which there is a
compatible legal history of the specification object; i.e.~it has precisely
the traces that correspond to histories that are synchronisation linearisable.
It is therefore enough to test whether the traces of the model of the
synchronisation object are a subset of the traces of \CSPM{Spec}, which can be
discharged using FDR.

We now generalise this approach.  For a synchronisation involving $k$ threads,
the corresponding \CSPM{sync} event contains $k$ thread identities,
$k$~parameters, and $k$~return values; each such event will be a
synchronisation (in the CSP model) between $k$ threads and the specification
process. 

For homogeneous synchronisations the identities of the threads (and
corresponding parameters and return values) may appear in either order within
the |sync| events.  The following definition of the lineariser allows this. 
%
\begin{cspm}
Lineariser(t) = 
  call.t.op?x -> (
    sync.t?t'!x?x'?y?y' -> return.t.op.y -> Lineariser(t)
    []
    sync?t'!t?x'!x?y'!y -> return.t.op.y -> Lineariser(t)
  )
\end{cspm}

Finally, for synchronisation objects with multiple synchronisation modes, the
specification process should have a different branch (with different
\CSPM{sync} events) for each mode.
