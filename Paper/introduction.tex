
\section{Introduction}

A common step of many concurrent programs involves two or more threads
\emph{synchronising}: each thread waits until other relevant threads have
reached the synchronisation point before continuing; in addition, the threads
can exchange data.  Reasoning about programs can be easier when
synchronisations are used: it helps us to reason about the states that
different threads are in. 

We study synchronisations in this paper: we formalise the requirements of
synchronisations, and describe testing and analysis techniques.

We start by giving some examples of synchronisations in order to illustrate
the idea.  (We use Scala notation; we explain non-standard aspects of the
language in footnotes.)  In each case, the synchronisation is mediated by a
\emph{synchronisation object}.

Perhaps the most common form of synchronisation object is a synchronous
channel.  Such a channel might have signature\footnote{The type {\scalashape
    Unit} is the type that contains a single value, the \emph{unit value},
  denoted~{\scalashape ()}.}
%
\begin{scala}
class SyncChan{
  def send(x: A): Unit
  def receive(): A
}
\end{scala}
%
Each invocation of one of the operations must synchronise with an invocation
of the other operation: the two invocations must overlap in time.  If an
invocation |send(x)| synchronises with an invocation of |receive|, then the
|receive| returns~|x|.

Sometimes an invocation may synchronise with an invocation of the same
operation.  For example, an \emph{exchanger} has the following signature.
%
\begin{scala}
class Exchanger{
  def exchange(x: A): A
}
\end{scala}
%
When two threads call |exchange|, they each receive the value passed in by the
other.  When invocations of two different operations synchronise, we use the
term \emph{heterogeneous}; where two invocations of the same operation
synchronise, we use the term \emph{homogeneous}.  

For some synchronisation objects, synchronisations might involve more than two
threads.  For example, an object of the following class
%
\begin{scala}
class Barrier(n: Int){
  def sync(): Unit
}
\end{scala}
%
can be used to synchronise~|n| threads, known as a \emph{barrier
  synchronisation}: each thread calls |sync|, and no invocation returns until
all~|n| have called it.

A \emph{combining barrier} also allows each thread to submit a parameter, and
for all to receive back some function of those parameters.\footnote{The Scala
  type {\scalastyle (A,A) =}$>$ {\scalastyle A} represents functions from
  pairs of {\scalastyle A} to~{\scalastyle A}.}
%
\begin{scala}
class CombiningBarrier(n: Int, f: (A,A) => A){
  def sync(x: A): A
}
\end{scala}
%
The function |f| is assumed to be associative.  If |n| threads call |sync|
with parameters $x_1, \ldots, x_{\ss n}$, in some order, then each receives
back $\sm f(x_1, \sm f(x_2, \ldots \sm f(x_{{\ss n}-1}, x_{\ss n}) \ldots ))$
(in the common case that |f| is commutative, this result is independent of the
order of the parameters).

In addition, we allow the synchronisations to be mediated by an object that
maintains some state between synchronisations.  As an example, consider a
synchronous channel that, in addition, maintains a sequence counter, and such
that both invocations receive the value of this counter.
\begin{scala}
class SyncChanCounter{
  private var counter: Int
  def send(x: A): Int
  def receive(): (A, Int)
}
\end{scala}

% Variations: homogenous case; different modes


Some synchronisation objects allow different modes of synchronisation.  For
example, consider a synchronous channel with timeouts: each invocation might
synchronise with another invocation, or might timeout without
synchronisation.  Such a channel might have a signature as follows.
%
\begin{scala}
class TimeoutChannel{
  def send(x: A): Boolean
  def receive(): Option[A]
}
\end{scala}
%
The |send| operation returns a boolean to indicate whether the send was
successful, i.e.~whether it synchronised.  The |receive| operation can return
a value |Some(x)| to indicate that it synchronised and received~|x|, or can
return the value |None| to indicate that it failed to synchronise\footnote{The
  type {\scalashape Option[A]} contains the union of such values.}.  Thus an
invocation of each operation may or may not synchronise with an invocation of
the other operation. 

A \emph{termination-detecting queue} can also be thought of as a stateful
synchronisation object with multiple modes.  Such an object acts like a
standard partial concurrent queue: if a thread attempts to dequeue, but the
queue is empty, it blocks until the queue becomes non-empty.  However, if a
state is reached where all the threads are blocked in this way, then they all
return a special value to indicate this fact.  In many concurrent algorithms,
such as a concurrent graph search, this latter outcome indicates that the
algorithm should terminate.  Such a termination-detecting queue might have the
following signature, where a dequeue returns the value |None| to indicate the
termination case.
%
\begin{scala}
class TerminationDetectingQueue(n: Int){ // £n£ is the number of threads   
  def enqueue(x: A): Unit
  def dequeue: Option[A]
}
\end{scala} 
%
The termination outcome can be seen as a synchronisation between all |n|
threads.  This termination-detecting queue combines the functionality of a
concurrent datatype and a synchronisation object.


%% In general, a synchronisation will involve some number~$k$ of threads, calling
%% operations of the form
%% %
%% \begin{scala}
%%   def op£\s1£(x£\s1£: A£\s1£): B£\s1£
%%   ...
%%   def op£\s k£(x£\s k£: A£\s k£): B£\s k£
%% \end{scala}
%% %
%% Each thread passes in some data, and receives back a result. 

In this paper, we consider what it means for one of these synchronisation
objects to be correct, and techniques for testing correctness.  

In Section~\ref{sec:spec} we describe how to specify a synchronisation
object.  The definition has similarities with the standard definition of
\emph{linearisation} for concurrent datatypes, except it talks about
synchronisations between invocations, rather than single invocations: we call
the property \emph{synchronisation linearisation}. 

In Section~\ref{sec:relating} we consider the relationship between
synchronisation linearisation and (standard) linearisation.  We show that the
two notions are different; but we show that synchronisation linearisation
corresponds to a small adaptation of linearisation, where an
operation of the synchronisation object may correspond to \emph{two} operations
of the object used to specify linearisation.  

We then consider testing of synchronisation object implementations.  Our
techniques are based on the techniques for testing (standard)
linearisation~\cite{gavin:lin-testing}, which we sketch in
Section~\ref{sec:lin-testing}.  In Section~\ref{sec:testing-hacking} we show
how the technique can be adapted to test for synchronisation linearisation,
using the result of Section~\ref{sec:relating}.  Then in
Section~\ref{sec:direct} we show how synchronisation linearisation can be
tested more directly, and present various complexity results.

In Section~\ref{sec:modelChecking} we consider how the property of
synchronisation linearisation can be analysed via model checking.  

