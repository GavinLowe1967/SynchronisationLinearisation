
\subsection{Case with state}

Suppose the specification object has non-trivial state. 

I think it will be more efficient to give a more direct implementation.
Define a configuration to be: (1)~a point in the log reached so far; (2)~the
set of pending operation invocations that have not synchronised; (3)~the set
of pending operation invocations that have synchronised (but not returned);
and (4)~the state of the sequential synchronisation object.  In any
configuration, can: synchronise a pair of pending operations (and update the
synchronisation object); advance in the log if the next event is a return that
is not pending; or advance in the log if the next event is a call.  Then
perform DFS.

Partial order reduction: a synchronisation point must follow either the
call of one of the concurrent operations, or another synchronisation
point.  Any synchronisation history can be transformed into this form, by
moving synchronisation points earlier, but not before any of the corresponding
call events, and preserving the order of synchronisations.  This means that
after advancing past the call of an invocation, we may synchronise that
invocation, and then an arbitrary sequence of other invocations. 

Alternatively, a synchronisation point must precede either the return of one
of the concurrent operations, or another synchronisation point.  This is more
like the JIT technique in the linearisability testing paper.  This means that
before advancing in the log to the return of an invocation that has not
synchronised, we synchronise some invocations, ending with the one in
question.  And we only synchronise in these circumstances. 

My intuition is that the former is more efficient: in the latter, we might
investigate synchronising other invocations even though the returning
operation can't be synchronised with any invocation.  

%%%%%

\subsubsection*{Complexity}

Consider the problem of testing whether a given concurrent history has
synchronisations consistent with a given sequential specification object. 

We make use of a result from~\cite{???} concerning the complexity of the
corresponding problem for linearisability.  Let |Variable| be a
linearisability specification object corresponding to a variable with |get|
and |set| operations.  Then the problem of deciding whether a given concurrent
history is linearisable with respect to |Variable| is NP-complete.

Let |ConcVariable| be a concurrent object that represents a variable.  

We consider concurrent synchronisation histories on an object with the
following signature.   
\begin{scala}
object VariableSync{
  def op£\s1£(op: String, x: Int): Int
  def op£\s2£(u: Unit): Unit
} 
\end{scala}
%
The intention is that |op|\s1|("get", x)| acts like |get(x)|, and
|op|\s1|("set", x)| acts like |set(x)| (but returns -1).  The |op|\s2
invocations do nothing except synchronise.  This can be captured formally by
the following synchronisation specification object.

\begin{scala}
object VariableSyncSpec{
  private var state = 0
  def sync((op, x): (String, Int), u: Unit): (Int, Unit) = 
    if(op == "get") (state, ()) else{ state = x; (-1, ()) }
}
\end{scala}


Let |ConcVariable| be a concurrent object that represents a variable.  Given a
concurrent history~$h$ of |ConcVariable|, we build a concurrent history~$h'$
of |VaraibleSync| as follows.  We replace every call or return of |get(x)| by
(respectively) a call or return of |op|\s1|("get", x)|; and we do similarly
with |set|s.  If there are $k$ calls of |get| or |set| in total, we prepend
$k$ calls of |op|\s2, and append $k$ corresponding returns (in any order).
Then it is clear that $h$ is linearisable with respect to |Variable| if and
only if $h'$ is linearisable with respect to |VariableSyncSpec|.

%%%%%%%%%%%%%%%%%%%%%%%%%%%%%%%%%%%%%%%%%%%%%%%%%%%%%%%%%%%%

\subsection{Stateless case}

In the stateless case, a completely different algorithm is possible.  Define
two invocations to be compatible if they could be synchronised, i.e.~they
overlap and the return values agree with those for the specification object.
For $n$ invocations of each operation (so a history of length~$4n$), this can
be calculated in $O(n^2)$.  Then find if there is a total matching in the
corresponding bipartite graph, using the Ford-Fulkerson method, which is
$O(n^2)$.
