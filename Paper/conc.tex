\section{Conclusions}
\label{sec:conc}

In this paper we have studied synchronisation objects.  We have identified
synchronisation linearisation as the appropriate safety condition, and
presented a way to specify allowed synchronisations, via synchronisation
specification objects.  We have proposed the liveness condition of
synchronisation progressibility.  We have shown how to reduce synchronisation
linearisation to standard synchronisation, where some operations are
linearised in two steps.

We have studied how to carry our testing on implementations
of synchronisation objects.  The approach is effective: the testing code is
easy to write; and the testers normally find errors quickly.  We have also
studied the complexity of algorithms for deciding whether a history is
synchronisation linearisable. 

Our approach assumes that each operation execution involves at most one
synchronisation.  We describe here a synchronisation object that does not fit
this pattern. 
%
In our implementation of a barrier synchronisation object~\cite{SCL}, threads
are arranged into a binary heap, with a |Signal| object, following the
signature in Figure~\ref{fig:Signal}, between each parent and child.  The
child thread calls |signalUpAndWait|, and the parent thread alternates between
calls of |waitForSignalUp| and |signalDown|.  Each call to |waitForSignalUp|
blocks until the child calls |signalUpAndWait|; and each call to
|signalUpAndWait| blocks until the parent calls |signalDown|.  Hence
|signal|\-|Up|\-|And|\-|Wait| synchronises with \emph{both} of the parent's
invocations.  Figure~\ref{fig:Signal} gives a timeline illustrating two
synchronisations.  
%% In the barrier object, the |Signal| objects are used to
%% send signals up the heap, each indicating that all the threads in the relevant
%% subheap have called the |sync| operation, and then to send signals back down
%% the heap, to indicate that all calls of |sync| can return.
%
We believe that we could adapt the notion of interval
linearisation~\cite{CRR-2018}, described in the next subsection, to our style
of specification using a specification object and abstraction function, and
also extend our testing algorithms to such cases.

%%%%%

\begin{figure}
\begin{scala}
  class Signal{
    def signalUpAndWait(): Unit
    def waitForSignalUp(): Unit
    def signalDown(): Unit
  }
\end{scala}
\def\eSet#1{node{$\set{\begin{align}#1}\end{align}$}}
%%%%%
\begin{center}
\begin{tikzpicture}[xscale = 0.9]
\draw(-0.8,0) node[left]{child:}; \draw(-0.8,-1) node[left]{parent:};
\draw[|-|] (0,0) -- node[above] {\scalastyle signalUpAndWait} (9.0,0);
\draw[|-|] (-0.5,-1) -- node[above] {\scalastyle waitForSignalUp} (3.5,-1);
\draw[|-|] (5,-1) -- node[above] {\scalastyle signalDown} (8.5,-1);
\draw (1.75,0) \X; \draw (1.75,-1) \X; \draw[dashed] (1.75,0) -- (1.75,-1);
\draw (6.75,0) \X; \draw (6.75,-1) \X; \draw[dashed] (6.75,0) -- (6.75,-1);
%% \draw (0.5,-2) \X;
%% \draw (0.5,-2.5) \eSet{\call.\sm{sUAW}, \\ \call.\sm{wFSU}};
%% \draw (3.0,-2) \X; \draw (3.0,-2.5) \eSet{\return.\sm{wFSU}};
%% \draw (5.5,-2) \X; \draw (5.5,-2.5) \eSet{\call.\sm{sD}};
%% \draw (8.0,-2) \X; 
%% \draw (8.0,-2.5)  \eSet{\return.\sm{sD},\\ \return.\sm{sUAW}};
\end{tikzpicture}
\end{center}
\caption{The signature of a {\scalashape Signal} object (top); and a timeline for
  a history (bottom).}
%% ; and a corresponding interval-sequential history (with
%%   names abbreviated) (bottom).}
\label{fig:Signal}
\end{figure}

%%%%%%%%%%%%%%%%%%%%%%%%%%%%%%%%%%%%%%%%%%%%%%%%%%%%%%%

\subsection{Related work}
\label{sec:related}

\paragraph{Variations of linearisation}

%% As noted in the Introduction, our notion of synchronisation linearisation is
%% equivalent to \emph{set linearisation}~\cite{Neiger-1994}, although no formal
%% definition was given.  The same property was rediscovered by Hemed et
%% al.~\cite{HRV-2015}, and called \emph{concurrency-aware linearisation} (CAL).
%% They define a concurrency-aware trace (CA-trace) to be a history of
%% \emph{sets} of operation executions, where each set represents operation
%% executions that seem to take effect simultaneously (such as those in a
%% synchronisation), and define CAL in terms of those CA traces.  They assume a
%% given set of correct CA-traces, rather than allowing an independent
%% specification, such as with the synchronisation specification objects we use.

As noted earlier, synchronisation linearisation is equivalent to Neiger's set
linearisation~\cite{Neiger-1994}, and Hemed et al.'s concurrency-aware
linearisation~\cite{HRV-2015}.  We review here a few other variants of
linearisation. 

%% Hemed et al.\ verify an exchanger object with
%% respect to CAL, and then use this result in a modular verification of an
%% elimination stack~\cite{HSY-2004}.

Casta\~neda et al.\ define \emph{interval linearisation} \cite{CRR-2023,
  CRR-2018} as a further generalisation of set linearisation.  This allows
them to capture the behaviour of objects where a particular operation
execution can interact at multiple points with other operations, for example
being involved in multiple synchronisations, like the |Signal| object
of Figure~\ref{fig:Signal}.
%
An \emph{interval-sequential history} is an alternating
sequence of sets of \call\ events and sets of \return\ events, where each
\return\ event matches a previous \call\ event.   The following is an
interval-sequential history corresponding to the timeline in
Figure~\ref{fig:Signal}. 
\[
\seq{
\begin{align}
  \set{\call.\sm{signalUpAndWait}, \call.\sm{waitForSignalUp}},\;
  \set{\return.\sm{waitForSignalUp}}, \\
  \set{\call.\sm{signalDown}},\;
  \set{\return.\sm{signalDown}, \return.\sm{signalUpAndWait}}
}.
\end{align}
\]
%% Figure~\ref{fig:Signal} gives an interval-sequential history for a |Signal|
%% object.
An \emph{interval-sequential specification} is a prefix-closed set of
interval-sequential histories.  An interval-sequential history~$h_s$
\emph{respects} a history~$h$ of the concurrent object if, for all operations
$op_1$ and~$op_2$, if $\return.op_1$ is before $\call.op_2$ in~$h$, then
in~$h_s$, every set containing $\call.op_1$ or $\return.op_1$ is before every
set containing $\call.op_2$ or $\return.op_2$.  The definition of interval
linearisation then follows in the obvious way. 
%%  It would be interesting to adapt the definition of interval linearisation
%% to our style of specification using a specification object and abstraction
%% function, and then to similarly adapt our testing algorithms.

%% Casta\~neda et al.~\cite{CRR-2018} prove a completeness result for interval
%% linearisation: Let $Spec$ be a non-empty prefix-closed set of well formed
%% histories, defining the histories of a concurrent object that are considered
%% correct; then $Spec$ can be captured as an interval-linearisability
%% specification.
%
%% Goubault, E., Ledent, J., and Mimram, S. Concurrent specifications beyond
%% linearizability. In Proceedings of. 22nd Intern Conf. on Principles of
%% Distributed Systems. LIPICS 125 (2018), 28:1–28:16.  Generalises this result.


% \framebox{Scherer and Scott}
Scherer et al.~\cite{SLS-2009} consider a setting where an operation may be
linearised in several steps, similar to our two-step linearisation.  Their
interest is in partial concurrent datatypes, where an operation may have a
nontrivial precondition.  Such an operation is linearised by (1)~an initial
request (where the operation registers itself); (2)~some number of
unsuccessful follow-ups (where the precondition is found not to be satisfied);
(3)~a successful follow-up (where the precondition holds, and the operation
takes effect).

Intermediate-value linearisability~\cite{IV-lin} is a relaxation of
linearisation for objects where queries return results from an ordered domain.
It requires that every history~$h$ must lie between two histories~$h_1$
and~$h_2$ of the specification object, with corresponding sequences of call
and return events, but where the returned values may be different: every value
returned by a query in~$h$ must be between the values returned by the
corresponding queries in~$h_1$ and~$h_2$.

% Refs 1, 4 from previous -- probably not.

Local linearisability~\cite{local-lin} concerns container objects with
disjoint operations to insert values, remove values, or observe values in the
container; the condition requires that, for each thread~$t$, the restriction
of a history to insertion operations of~$t$, and removal or observation
operations of values inserted by~$t$ is linearisable with respect to a
suitable sequential specification.

Linearisability has been extended to deal with system crashes in a number of
ways, including strict linearisability~\cite{strict-lin}, nesting-safe
recoverable linearisability~\cite{recoverable-lin}, and durable
linearisability~\cite{durable-lin}. 

%% Strict linearisability~\cite{strict-lin} limits the effects of an operation
%% that crashes.  % ref~1 of \cite{CRR-2023}.  

% Nesting-safe recoverable
%% linearisability~\cite{recoverable-lin} extends linearisation to consider
%% the case where an operation crashes and is subsequently recovered.   ref 4
%% of \cite{CRR-2023}.  https://dl.acm.org/doi/pdf/10.1145/3212734.3212753

%% Durable linearisability: ref 25 of \cite{CRR-2023}.
%% https://www.cs.rochester.edu/u/scott/papers/2016_DISC_persistence.pdf

%%%%%

\paragraph{Testing}

We are not aware of any prior work testing the correctness of synchronisation
objects.  
%
Linearisation testing has been described previously
in~\cite{wing-gong,gavin:lin-testing}.  Specific implementations include
Knossos~\cite{knossos} and Porcupine~\cite{porcupine}.


%%%%%

\paragraph{Verification}

Our analysis technique in this paper has been software testing of
implementations of synchronisation objects.  However, one can also apply model
checking to the problem.  The companion paper~\cite{gavin:SCL-CSP} analyses a
library of communication primitives (including a closeable channel with timed
operations, and alternation), using CSP~\cite{awr:ucs} and its model checker
FDR~\cite{fdr3}.  Synchronisation linearisation is specified via a
generic way of capturing the traces (equivalent to histories in this paper)
that satisfy the property.  Synchronisation linearisation is a safety
property, so is tested in the traces model.  It turns out that testing against
the same specification, but in the failures-divergences model, captures
synchronisation progressibility.  The library includes several different
synchronisation objects that are analysed in this way.
An error is identified on a previous version of the library;
but the revised version is shown to be synchronisation linearisable and
progressible.

There have been many approaches to verifying standard linearisation,
e.g.~\cite{HSV13, CHSV2015, DHK15, BEEH18, AHHJR}.  Dongol and
Derrick~\cite{dongol-derrick} survey much of this work.  It would be
interesting to try to adapt some of these techniques to synchronisation
linearisation.

%%%%%%%%%%%%%%%%%%%%%%%%%%%%%%%%%%%%%%%%%%%%%%%%%%%%%%%

\subsection*{Acknowledgements}

We would like to thank the anonymous reviewers for their useful comments and
suggestions. 

\bibliographystyle{elsarticle-num}
%\bibliographystyle{alpha}
\bibliography{sync}
