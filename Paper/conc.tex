\section{Conclusions}
\label{sec:conc}

In this paper we have studied synchronisation objects.  We have proposed the
correctness conditions of synchronisation linearisation and synchronisation
progressibility.  We have studied how to carry our testing on implementations
of synchronisation objects.  The approach is effective: the testing code is
easy to write; and the testers normally find errors quickly.  We have also
studied the complexity of algorithms for deciding whether a history is
synchronisation linearisable. 

Our analysis technique in this paper has been software testing of
implementations of synchronisation objects.  However, one can also apply model
checking to the problem.  The companion paper~\cite{gavin:SCL-CSP} analyses a
library of communication primitives (including a closeable channel with timed
operations, and alternation), using CSP~\cite{awr:ucs} and its model checker
FDR~\cite{fdr3}.  An error is identified on a previous version of the library;
but the revised version is shown to be synchronisation linearisable and
progressible.

\bibliographystyle{elsarticle-num}
%\bibliographystyle{alpha}
\bibliography{sync}
