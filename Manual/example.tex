
\section{An example script}

We introduce the technique of writing testing scripts via a simple example.
Consider a class |SyncChan| that implements a synchronous channel, extending a
trait~|Chan|, defined as follows.
%
\begin{scala}
trait Chan[A]{
  def !(x: A): Unit
  def ?(u:Unit): A
}

class SyncChan[A] extends Chan[A]{ ... }
\end{scala}
%
The intention is that an invocation |!x| sends the value~|x|, and syncronises
with an invocation |?()|, which should return~|x|.

%%%%%

\begin{figure}
\begin{scala}
object ChanTester{
  // Representation of operations within the log
  trait Op
  case class Send(x: Int) extends Op
  case object Receive extends Op

  /** A worker.  An even number of these workers should not produce a
    * deadlock. */
  def worker(c: Chan[Int])(me: Int, log: HistoryLog[Op]) = {
    for(i <- 0 until 20)
      if(me%2 == 0) log(me, c?(), Receive)
      else{ val x = Random.nextInt(100); log(me, c!x, Send(x)) }
  }

  /** The specification class. */
  object Spec{
    def sync(x: Int, u: Unit) = ((), x)
  } 

  /** Mapping showing how synchronisations of concrete operations correspond
    * to operations of the specification object. */
  def matching: PartialFunction[(Op,Op), (Any,Any)] = {
    case (Send(x), Receive) => Spec.sync(x, ()) 
  }

  /** Do a single test.  Return true if it passes. */
  def doTest: Boolean = {
    val c: Chan[Int] = new SyncChan[Int]
    val bst = new BinaryStatelessTester[Op](worker(c), 4, matching)
    bst()
  }

  def runTests(reps: Int) = {
    var i = 0; while(i < reps && doTest) i += 1
  }

  def main(args: Array[String]) = runTests(5000)
}
\end{scala}
\caption{An example testing script.\label{fig:script}}
\end{figure}

%%%%%

Figure~\ref{fig:script} gives a stripped-down script for testing objects of
this class for synchronisation linearisation.  (The full script allows several
different classes that implement |Chan| to be tested, and replaces the numeric
constants in the script with variables that can be set on the command line;
the same is true of later example scripts.)

%% The |ChanTester| object extends a trait |Tester|, outlined in
%% Figure~\ref{fig:tester}.  Thus each testing object based on |Tester| must
%% define a function |doTest| that performs a single test, returning |true| if
%% the test passes.  The |Tester| trait then provides a function |runTests| that
%% executes |doTest| a given number of times, or until a test fails.

%%%%%%%%%%

%% \begin{figure}
%% \begin{scala}
%% /** Base class for testers, combining common code. */
%% trait Tester{
%%   /** Do a single test.  Return true if it passes.  Defined in subclasses.  */
%%   def doTest: Boolean

%%   /** Run `reps` tests.
%%     * @param timing are we doing timing experiments?
%%     * @param countReps are we doing an experiment counting the number of 
%%     * repetitions?  */
%%   def runTests(reps: Int, timing: Boolean = false, countReps: Boolean = false) = ...
%% }
%% \end{scala}
%% \caption{The {\scalashape Tester} trait.\label{fig:tester}}
%% \end{figure}

  %% /** Number of worker threads to run. */
  %% var p = 4

  %% /** Number of iterations per worker thread. */
  %% var iters = 20

  %% /** Do we check the progress condition? */
  %% var progressCheck = false

%%%%%%%%%%

%Going back to the |ChanTester| object, 

The trait |Op| gives a representation of operations.  The subclasses have
obvious meanings.  Objects of these subclasses are stored in the log, and so
correctness is defined in terms of them.

Each worker thread that performs invocations on a channel~|c| is defined by
the function |worker(c)|.  This function takes as parameters the identity~|me|
of the worker, and a log object~|log|.  Here, in order to avoid deadlocks, we
run an even number of workers, where workers with an even identity perform
receives, and workers with an odd identity perform sends, with 20~invocations
in each case.  The code |log(me, c?(), Receive)| (a call of the |apply|
operation on the |log| object) performs a receive.  The three parameters are:
the identity of the thread; the invocation to be performed on the channel; and
the representation of the invocation to be used in the log.  This logs the
call of the invocation, performs the invocation on the channel, and logs the
return together with the result of the invocation (i.e.~the value received).
Similarly, each worker with an odd identity sends a random value~|x|, with
suitable logging, via the code |log(me, c!x, Send(x)|.

The specification of the channel is captured by the combination of the
synchronisation specification object |Spec| and the partial
function~|matching|.  The latter captures that invocations represented in the
log by |Send(x)| and |Receive| can synchronise together, and the values each
returns are given by |Spec.sync(x, ()) = ((), x)|; i.e.~the send should return
the unit value~|()|, and the receive should return~|x|.  More generally, the
domain of the |matching| function represents all pairs of invocations that can
synchronise.  (Alternatively, we could dispense with |Spec|, and inline its
|sync| method; however, it can be more convenient to have an explicit
synchronisation specification object, particularly in the case of stateful
objects.)

The function |doTest| performs a single test, and returns a boolean indicating
if the resulting history is synchronisation linearisable.  It creates a
particular |SyncChan| object~|c| to be tested.  The class
|BinaryStatelessTester| encapsulates an algorithm for testing a binary
heterogeneous stateless synchronisation object, such as these channels.  The
constructor takes: a type parameter corresponding to the log representation of
invocations (here |Op|); a function representing a single worker (here
|worker(c)|); the number of workers to run (here~|4|); and the partial
function that specifies the synchronisations (here~|matching|).  The |doTest|
function creates a tester object~|bst|.  The code |bst()| (a call to the
|apply| method of~|bst|) then runs the worker threads, and tests whether the
resulting log is synchronisation linearisable; the result of that call is also
the result of~|doTest|.

The |runTests(reps)| executes |doTest| at most |reps| times, or until it finds
an erroneous history.  Finally, the |main| method uses the |runTests| method
to perform 5000 tests.

%%%%%%%%%%%%%%%%%%%%%%%%%%%%%%%%%%%%%%%%%%%%%%%%%%%%%%%

\subsection{Pragmatics}

When running a tester, it is useful to see some indication that it is making
progress, and in particular that the threads working on the synchronisation
object have not deadlocked.  One way is to print something on the screen
occasionally, say printing a dot every 100 tests.

The trait |synchronisationTester.Tester| in the distribution, outlined below,
encapsulates some boiler-plate code often used in testers.
%
\begin{scala}
trait Tester{
  /** Number of worker threads to run. */
  var p = 4

  /** Number of iterations per worker thread. */
  var iters = 20

  /** Do we check the progress condition? */
  var progressCheck = false

  /** Do a single test.  Return true if it passes.  Defined in subclasses.  */
  def doTest: Boolean

  /** Run `reps` tests.
    * @param timing are we doing timing experiments? */
  def runTests(reps: Int, timing: Boolean = false) = { ... }
}
\end{scala}
%
Objects that extend |Tester| must provide the |doTest| method, but can inherit
the |runTest| method, which executes |doTest| a set number of times or until
an error is found.  Thus we could have extended |Tester| in
Figure~\ref{fig:script}, and avoided defining~|runTests|.  The |runTests|
method also prints dots to show progress, as described above, by default, one
dot every 100 runs (if the |progressCheck| flag is set, it prints a dot after
every test; this is intended for use with progress checks, described below,
which tend to be slower). 

In addition, the |doTest| method prints the time taken, in milliseconds.  If
the optional |timing| parameter is set, the time is printed in nanoseconds;
this is intended for use in timing experiments.

It is useful to allow various parameters of tests to be set on the command
line, such as: the number of worker threads to use in each test; the number of
iterations to be performed by each worker in each test; the number of tests to
perform; and the maximum value to use for a data value (such as the value sent
in Figure~\ref{fig:script}).  The variables |p| and |iters| in |Tester| are
intended to be used for the first two of these.

%%%%%%%%%%

\subsection{Progress checks}

In order to check for synchronisation progressibility (in addition to
synchronisation linearisation), it is necessary to pass a positive integer
value to the apply function of~|bst|, representing a timeout time, in
milliseconds.  For example
%
\begin{scala}
  bst(100)
\end{scala}
%
This will run threads, but interrupt them after the specified time, in
milliseconds, if they have not all terminated.  It then tests the resulting
log for synchronisation progressibility, i.e.~checking that no pending
invocations failed to return when they could have done.

Each of the other testing algorithms, described later, also has an |apply|
function that takes an optional integer argument, with the same meaning. 

It is necessary to choose a timeout time that is large enough to ensure that
any threads that can still run have time to do so, i.e.~to avoid interrupting
threads that were about to return, which would lead to a false failure of
progressibility being reported.  Conversely, too large a timeout time will
make the testing take longer.  Our experience is that a time of 100ms is
suitable on most synchronisation objects.

When checking for progressibility, it is no longer necessary to design the
threads to avoid deadlocks.  Indeed, it is sensible to allow the possibility
of deadlocks, in order to provide greater test coverage.  A sensible approach
is to arrange for workers to pick invocations at random, for example:
%
\begin{scala}
  def worker(c: Chan[Int])(me: Int, log: HistoryLog[Op]) = {
    for(i <- 0 until 20)
      if(Random.nextInt(2) == 0) log(me, c?(), Receive)
      else{ val x = Random.nextInt(100); log(me, c!x, Send(x)) }
  }
\end{scala}

The interruption is done by calling the |interrupt| method of the |Thread|
class on each worker, expecting each to throw an |InterruptedException|.
Most, but not all, concurrency primitives will react to the |interrupt| method
as required.  In some cases, it will be necessary to build in an additional
mechanism to deal with the interruption.   


%%%%%%%%%%%%%%%%%%%%%%%%%%%%%%%%%%%%%%%%%%%%%%%%%%%%%%%
