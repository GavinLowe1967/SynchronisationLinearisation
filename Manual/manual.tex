\documentclass[12pt,a4paper]{article}

\usepackage{scalalistings}
\scalaMid

\title{Testing Synchronisation Objects: User Manual}
\author{Gavin Lowe}
 
\sloppy
\def\topfraction{0.8}
\renewcommand{\floatpagefraction}{0.75}

\def\upto#1#2{{[#1\mathbin{..}#2)}} 

\begin{document}
\maketitle

\section{Introduction}

This user manual describes a package for testing synchronisation objects.  It
should be read in conjunction with the paper \emph{Understanding
  Synchronisation}~\cite{sync}.  

worker threads
logging.

pragmatics -- shortish runs

assume familiarity with Scala

install, using.

\framebox{Overview, roadmap}

Possible order:
Example script;

Interpreting errors.

Variations: 
Progress
Different testing algorithms -- after progress; 
Stateful; 
Progress checks


Interpreting errors;
Fine details? ; 
Tactics (params in Tester).  Avoid deadlocks -- single invocation. 


%%%%%%%%%%%%%%%%%%%%%%%%%%%%%%%%%%%%%%%%%%%%%%%%%%%%%%%

\section{An example script}
\label{sec:example}

We outline how to write a testing script via a simple example.  Consider a
class |SyncChan| that implements a synchronous channel, extending a
trait~|Chan|, defined as follows.
%
\begin{scala}
trait Chan[A]{
  def send(x: A): Unit
  def receive(): A
}

class SyncChan[A] extends Chan[A]{ ... }
\end{scala}
%
The intention is that an invocation |send(x)| sends the value~|x|, and
syncronises with an invocation |receive()|, which should return~|x|.

%%%%%

\begin{figure}
\begin{scala}
object ChanTester{
  // Representation of operations within the log
  trait Op
  case class Send(x: Int) extends Op
  case object Receive extends Op

  /** A worker.  An even number of these workers should not produce a
    * deadlock. */
  def worker(c: Chan[Int])(me: Int, log: HistoryLog[Op]) = {
    for(i <- 0 until 4)
      if(me%2 == 0) log(me, c.receive(), Receive)
      else{ val x = Random.nextInt(100); log(me, c.send(x), Send(x)) }
  }

  /** The specification class. */
  object Spec{
    def sync(x: Int, u: Unit) = ((), x) 
  } 

  /** Mapping showing how synchronisations of concrete operations correspond
    * to operations of the specification object. */
  def matching: PartialFunction[(Op,Op), (Any,Any)] = {
    case (Send(x), Receive) => Spec.sync(x, ())   // £= ((), x)£
  }

  /** Do a single test.  Return true if it passes. */
  def doTest: Boolean = {
    val c: Chan[Int] = new SyncChan[Int]
    val bst = new BinaryStatelessTester[Op](worker(c), 4, matching)
    bst()
  }

  def runTests(reps: Int) = {
    var i = 0; while(i < reps && doTest) i += 1
  }

  def main(args: Array[String]) = runTests(5000)
}
\end{scala}
\caption{An example testing script.\label{fig:script}}
\end{figure}

%%%%%

Figure~\ref{fig:script} gives a stripped-down script for testing objects of
this class for synchronisation linearisation.  (The full script allows several
different classes that implement |Chan| to be tested, and replaces the numeric
constants in the script with variables that can be set on the command line;
the same is true of later example scripts.)

%% The |ChanTester| object extends a trait |Tester|, outlined in
%% Figure~\ref{fig:tester}.  Thus each testing object based on |Tester| must
%% define a function |doTest| that performs a single test, returning |true| if
%% the test passes.  The |Tester| trait then provides a function |runTests| that
%% executes |doTest| a given number of times, or until a test fails.

%%%%%%%%%%

%% \begin{figure}
%% \begin{scala}
%% /** Base class for testers, combining common code. */
%% trait Tester{
%%   /** Do a single test.  Return true if it passes.  Defined in subclasses.  */
%%   def doTest: Boolean

%%   /** Run `reps` tests.
%%     * @param timing are we doing timing experiments?
%%     * @param countReps are we doing an experiment counting the number of 
%%     * repetitions?  */
%%   def runTests(reps: Int, timing: Boolean = false, countReps: Boolean = false) = ...
%% }
%% \end{scala}
%% \caption{The {\scalashape Tester} trait.\label{fig:tester}}
%% \end{figure}

  %% /** Number of worker threads to run. */
  %% var p = 4

  %% /** Number of iterations per worker thread. */
  %% var iters = 20

  %% /** Do we check the progress condition? */
  %% var progressCheck = false

%%%%%%%%%%

%Going back to the |ChanTester| object, 

The trait |Op| gives a representation of operation invocations.  The
subclasses have obvious meanings.  Objects of these subclasses are stored in
the log, and so correctness is defined in terms of them.

Each worker thread that performs invocations on a channel~|c| is defined by
the function |worker(c)|.  This function takes as parameters the identity~|me|
of the worker, and a log object~|log|.  Here, in order to avoid deadlocks, we
run an even number of workers, where workers with an even identity perform
receives, and workers with an odd identity perform sends.  Each worker
performs 4~invocations.  

The code |log(me, c.receive(), Receive)| (a call of the |apply| operation on the
|log| object) performs a receive.  The three parameters are: the identity of
the thread; the invocation to be performed on the channel; and the
representation of the invocation to be used in the log.  This logs the call of
the invocation, performs the invocation on the channel, and logs the return
together with the result of the invocation (i.e.~the value received).
Similarly, each worker with an odd identity sends a random value~|x|, with
suitable logging, via the code |log(me, c.send(x), Send(x))|.

The specification of the channel is captured by the combination of the
synchronisation specification object |Spec| and the partial
function~|matching|.  The latter captures that invocations represented in the
log by |Send(x)| and |Receive| can synchronise together, and the values each
returns are given by |Spec.sync(x, ()) = ((), x)|; i.e.~the send should return
the unit value~|()|, and the receive should return~|x|.  More generally, the
domain of the |matching| function represents all pairs of invocations that can
synchronise.  (Alternatively, we could dispense with |Spec|, and inline its
|sync| method; however, it can be more convenient to have an explicit
synchronisation specification object, particularly in the case of stateful
objects.)

The function |doTest| performs a single test, and returns a boolean indicating
if the resulting history is synchronisation linearisable.  It creates a
particular |SyncChan| object~|c| to be tested.  The class
|BinaryStatelessTester| encapsulates an algorithm for testing a binary
heterogeneous stateless synchronisation object, such as a synchronous channel.
The constructor takes: a~type parameter corresponding to the log
representation of invocations (here~|Op|); a~function representing a single
worker (here |worker(c)|); the number of workers to run (here~|4|); and the
partial function that specifies the synchronisations (here~|matching|).  The
|doTest| function creates a tester object~|bst|.  The code |bst()| (a call to
the |apply| method of~|bst|) then runs the worker threads, and tests whether
the resulting log is synchronisation linearisable; the result of that call is
also the result of~|doTest|.

Note that each test is rather short: four workers, each performing four
invocations.  Experiments in~\cite{sync} show that running short tests like
this tends to find errors, if they exist, faster.  

The code |runTests(reps)| executes |doTest| at most |reps| times, or until it
finds an erroneous history.  Finally, the |main| method uses the |runTests|
method to perform at most 5000 tests.

%%%%%%%%%%%%%%%%%%%%%%%%%%%%%%%%%%%%%%%%%%%%%%%%%%%%%%%

\subsection{Using the {\scalashape Tester} interface}

%% When running a tester, it is useful to see some indication that it is making
%% progress, and in particular that the threads working on the synchronisation
%% object have not deadlocked.  One way is to print something on the screen
%% occasionally, say printing a dot every 100 tests.

%%%%%

\begin{figure}
\begin{scala}
trait Tester{
  /** Number of worker threads to run. */
  var p = 4

  /** Number of iterations per worker thread. */
  var iters = 4

  /** Do we check the progress condition? */
  var progressCheck = false

  /** Number of runs for each dot printed. */
  def runsPerDot = if(progressCheck) 1 else 100

  /** Do a single test.  Return true if it passes.  Defined in subclasses.  */
  def doTest: Boolean

  /** Run `reps` tests.
    * @param timing are we doing timing experiments? */
  def runTests(reps: Int, timing: Boolean = false) = { ... }
}
\end{scala}
\caption{The {\scalashape Tester} interface.  \label{fig:tester}}
\end{figure}

%%%%%

The trait |synchronisationTester.Tester| in the distribution, outlined in
Figure~\ref{fig:tester}, encapsulates some boiler-plate code often used in
testers.  Objects that extend |Tester| must provide the |doTest| method, but
can inherit the |runTests| method, which executes |doTest| a set number of
times or until an error is found.  Thus we could have extended |Tester| in
Figure~\ref{fig:script}, and avoided defining~|runTests|.

The |runTests| method also prints dots on the screen to show progress.  This
can be useful to be sure that the threads working on the synchronisation
object have not deadlocked.  By default, |runTests| prints one dot every 100
runs, or, if the |progressCheck| flag is set, it prints a dot after every test
(this flag is intended for use with progress checks, described in
Section~\ref{sec:progress}, which tend to be slower); however, the frequency
of dots can be adjusted by overriding |runsPerDot|.

In addition, the |doTest| method prints the time taken, in milliseconds.  If
the optional |timing| parameter is set, the time is printed in nanoseconds;
this is intended for use in timing experiments.

It is useful to allow various parameters of tests to be set on the command
line, such as: the number of worker threads to use in each test; the number of
iterations to be performed by each worker in each test; the number of tests to
perform; and the maximum value to use for a data value (such as the value sent
in Figure~\ref{fig:script}).  The variables |p| and |iters| in |Tester| are
intended to be used for the first two of these.

%%%%%%%%%%

\subsection{Progress checks}
\label{sec:progress}

In order to check for synchronisation progressibility (in addition to
synchronisation linearisation), it is necessary to pass a positive integer
value to the apply function of~|bst|, representing a timeout time, in
milliseconds.  For example
%
\begin{scala}
  bst(100)
\end{scala}
%
This will run threads, but interrupt them after the specified time, in
milliseconds, if they have not all terminated.  It then tests the resulting
log for synchronisation progressibility, i.e.~checking that no pending
invocations failed to return when they could have done.

Each of the other testing algorithms, described later, also has an |apply|
function that takes an optional integer argument, with the same meaning. 

It is necessary to choose a timeout time that is large enough to ensure that
any threads that can still run have time to do so, i.e.~to avoid interrupting
threads that were about to return, which would lead to a false failure of
progressibility being reported.  Conversely, too large a timeout time will
make the testing take longer.  Our experience is that a time of 100ms is
suitable on most synchronisation objects.

When checking for progressibility, it is no longer necessary to design the
threads to avoid deadlocks.  Indeed, it is sensible to allow the possibility
of deadlocks, in order to provide greater test coverage.  A sensible approach
is to arrange for workers to pick invocations at random, for example:
%
\begin{scala}
  def worker(c: Chan[Int])(me: Int, log: HistoryLog[Op]) = {
    for(i <- 0 until 20)
      if(Random.nextInt(2) == 0) log(me, c.receive(), Receive)
      else{ val x = Random.nextInt(100); log(me, c.send(x), Send(x)) }
  }
\end{scala}

The interruption is done by calling the |interrupt| method of the |Thread|
class on each worker, expecting each to throw an |InterruptedException|.
Most, but not all, concurrency primitives will react to the |interrupt| method
as required.  In some cases, it will be necessary to build in an additional
mechanism to deal with the interruption.   


%%%%%%%%%%%%%%%%%%%%%%%%%%%%%%%%%%%%%%%%%%%%%%%%%%%%%%%

\section{Other algorithms}
\label{sec:algorithms}

In the previous section, we considered the class |BinaryStatelessTester| for
testing binary heterogeneous stateless synchronisation objects.  In this
section, we describe classes that encapsulate other testing algorithms.

All example scripts referred to in this section can be found in
Appendix~\ref{app:examples}.  Fuller versions of each script are included in
the distribution.

Each script tests only for synchronisation linearisation, but can be adapted
to test for progressibility by passing a positive integer to the |apply|
method of the object encapsulating the algorithm, as for the synchronous
channel tester in the previous section.

%%%%%%%%%%%%%%%%%%%%%%%%%%%%%%%%%%%%%%%%%%%%%%%%%%%%%%%

\subsection{Stateless testers}

Figure~\ref{fig:stateless-testers} gives the signatures for each of the
classes that encapsulates an algorithm for testing a stateless synchronisation
object (including |BinaryStatelessTester|, described in the previous section).

The class |HomogeneousBinaryStatelessTester| is for testing binary homogeneous
stateless synchronisation objects.  It takes the same parameters as in the
heterogeneous case; and likewise the |apply| function takes the same optional
parameter.

%%%%%%%%%%

\begin{figure}
\begin{scala}
/** Testing algorithm for binary heterogeneous stateless synchronisation
  * objects. */
class BinaryStatelessTester[Op](
    worker: (Int, HistoryLog[Op]) => Unit,
    p: Int,
    matching: PartialFunction[(Op,Op), (Any,Any)]
){
  def apply(delay: Int = -1): Boolean
}

/** Testing algorithm for binary homogeneous stateless synchronisation
  * objects. */
class HomogeneousBinaryStatelessTester[Op](
    worker: (Int, HistoryLog[Op]) => Unit,
    p: Int,
    matching: PartialFunction[(Op,Op), (Any,Any)]
){
  def apply(delay: Int = -1): Boolean
}

/** Testing algorithm for general binary stateless synchronisation objects. */
class StatelessTester[Op](  
    worker: (Int, HistoryLog[Op]) => Unit,
    p: Int, 
    arities: List[Int],
    matching: PartialFunction[List[Op], List[Any]],
    suffixMatching: List[Op] => Boolean = (es: List[Op]) => true
){
  def apply(delay: Int = -1): Boolean
}
\end{scala}
\caption{Signatures for the stateless testers.\label{fig:stateless-testers}}
\end{figure}

%%%%%%%%%%

Figure~\ref{fig:homo-binary} gives an example for this class, giving a testing
script for an exchanger.  Threads call the method |exchange| on the exchanger,
passing in a value; this call should synchronise with another call, and both
threads should receive the other's value.  

Most parts of the script are straightforward, and similar to the script for
the synchronous channel.  

One point to note about this script is that each worker thread performs a
\emph{single} invocation: otherwise it is possible for the system to deadlock,
for example if one thread has two remaining invocations but all the others
have terminated.  This tactic might be of use elsewhere.
 
%%%%%%%%%%%%%%%%%%%%%%%%%%%%%%%%%%%%%%%%%%%%%%%%%%%%%%%%%%%%%%%%%

The class |StatelessTester| can test arbitrary stateless synchronisation
objects.  Its signature is again in Figure~\ref{fig:stateless-testers}.
The parameters |Op|, |worker| and~|p| are as for the binary testers.  The
parameter |arities| is a list of all possible arities of synchronisations.
The parameter, |matching| is much like in the binary case, except its domain
is all \emph{lists} of operations that might synchronise, and it returns a
corresponding \emph{list} of expected results.  Finally, the optional
parameter |suffixMatching| is a function that should return |true| when its
argument is a suffix of a possible synchronisation (with default value that
always returns |true|); we explain below how this can be useful in
optimisations.

%%%%%

Figure~\ref{fig:ABC} gives an example, giving a testing script for the ABC
problem.  Here, the synchronisation object provides three operations,
|syncA(a: A)|, |syncB(b: B)|, and |syncC(c: C)|.  Each synchronisation should
be between three invocations, one of each operation, with each invocation
returning the parameters of the other two invocations.

Most aspects of the tester are straightforward.  Here the tester runs 6
threads (the number must be a multiple of~3 to avoid deadlocks), with two for
each operation.  Each synchronisation involves 3~invocations.

%%%%%

Figure~\ref{fig:timeout-chan} gives another example, for a timeout channel,
where an invocation can fail to synchronise and timeout.  The |sendWithin|
operation returns a boolean, indicating whether is correctly sent its value.
The |receiveWithin| operation returns an |Option| value, with |Some(x)|
indicating that it received~|x|, and |None| indicating that it timed out.

The definition of |matching| illustrates how to specify mixed modes of
synchronisation: a |send(x)| may fail to synchronise, so timeout and return
|false|; a |receive| may fail to synchronise, so timeout and return |None|; or
a |send(x)| and |receive| may synchronise and return |true| and |Some(x)|,
respectively.  Thus synchronisations may have arities~1 or~2, as captured by
the parameter |List(1,2)| of the |StatelessTester| constructor.

%%%%%%%%%%

Figure~\ref{fig:barrier}  gives an example of a
tester for a barrier synchronisation object.  Each such object is used by some
number~|n| of threads, each of which calls an operation~|sync|: no call to
|sync| should return until all~|n| threads have called it, so this
synchronises all |n|~threads.

Most aspects of the script are straightforward.  A synchronisation will be
represented by a list of |n| |Sync| objects, one for each worker.  However,
with a naive approach, each such synchronisation could be represented in
$\sm{n}!$ different ways, giving an increase in the complexity of checking.
We therefore make the decision that we will require each such list to be in
sorted order of the workers' identities (as tested by the recursive |isSorted|
function), so as to reduce the number of cases considered. 

We also supply an argument for the optional parameter |suffixMatching| of the
|StatelessTester| constructor.  Recall that this parameter tests whether its
argument is a suffix of a possible synchronisation, so this function tests if
the identities are of the form $\interval{k}{\sm n}$ for some~$k$.
Internally to the |StatelessTester|, this reduces the
number of lists of operations built as possible synchronisations.  When run
with six worker threads, the use of this parameter reduces the running time of
the tester by a factor if over 20, although the speed-up is less with fewer
workers.

 % other algorithms
\subsection{Stateful testers}

We now consider stateful testers.  

The algorithm for testing binary stateful synchronisation objects is
encapsulated in the class |BinaryStatefulTester| whose signature is in
Figure~\ref{fig:stateful-testers}.  The parameters are as follows.
%
\begin{itemize}
\item The type parameter |Op| is the representation of operation invocations
  in the log, as before.

\item The type parameter |S| is the type of synchronisation specification
  objects, giving an abtract representation of the synchronisation object.
  Such objects should be \emph{immutable}; they should have a suitable
  definition for equality (|equals|) and a compatible |hashCode|.

\item The parameter |worker| defines a worker that operates on the
  synchronisation object, as before.

\item The parameter~|p| gives the number of threads to run, as before.

\item The parameter |specMatching| captures the specification.  This function
  takes a parameter~|s| corresponding to the state of the specification
  object.  Then |specMatching(s)| is a partial function defining what
  synchronisations are allowed given the state~|s| for the specification
  object.  Its domain, as before, is the pairs of invocations that may
  synchronise; the function returns the resulting state of the specification
  object, and the results to be returned by the two invocations.

\item The parameter |spec0| is the initial state of the specification object.

\item The optional parameter |doASAP| specifies whether the ASAP partial order
  reduction should be employed.  Experience suggests that this is normally
  beneficial. 
\end{itemize}

%%%%%%%%%%

\begin{figure}
\begin{scala}
class BinaryStatefulTester[Op,S](
  worker: (Int, HistoryLog[Op]) => Unit,
  p: Int,
  specMatching: S => PartialFunction[(Op,Op), (S,(Any,Any))],
  spec0: S,
  doASAP: Boolean = false
){
  def apply(delay: Int = -1): Boolean
}

class StatefulTester[Op,S](
  worker: (Int, HistoryLog[Op]) => Unit,
  p: Int, 
  arities: List[Int],
  matching: S => PartialFunction[List[Op], (S,List[Any])],
  suffixMatching: List[Op] => Boolean = (es: List[Op]) => true,
  spec0: S, 
  doASAP: Boolean = false
){
  def apply(delay: Int = -1): Boolean
}
\end{scala}
\caption{Signatures for the stateful testers.\label{fig:stateful-testers}}
\end{figure}

%

%%%%%%%%%%%%%%%%%%%%%%%%%%%%%%%%%%%%%%%%%%%%%%%%%%%%%%%

%%%%%

Figures~\ref{fig:one-family-1} and~\ref{fig:one-family-2} give an example, for
the one-family problem.  Here, |n| threads, with identities $\upto0{\sm{n}}$,
each call a method~|sync| at most $\sm{n}-1$ times, passing in its own
identity.  Each such invocation should synchronise with another invocation,
and return the identity of the other thread; however, each pair of threads
should synchronise together at most once.  Hence the synchronistion object is
stateful: abstractly, its state is the set of pairs of threads that have
synchronised so far.

The specification object |Spec(bitMap)| captures this state using the
bitmap~|bitMap|; for each pair of threads~|a| and |b|,\, |bitMap(a)(b)| is
true if they have already synchronised.  The operation |sync(a, b)| specifies
the result of a synchronisation between threads~|a| and~|b|.  This is allowed
only if |a| and |b| have not already synchronised, as captured by the
|requires| check; if this check fails, the testing framework catches the
resulting |Illegal|\-|Argument|\-|Exception|, but does not allow the
synchronisation to be linearised in this state.  If the synchronisation is
allowed, it creates a new bitmap recording the synchronisation, and returns a
corresponding new specification object, together with the correct results for
the two invocations.  Recall that the specification object must be immutable:
hence we create a new specification object rather than simply updating the
current one.  Recall also that the specification object must have appropriate
definitions of equality and hash code: we define equality as value equality
over the bitmaps, and the hash code based directly on the content of the
bitmap.

The |matching| function defines that two invocations may synchronise as
captured by the |sync| method on the current specification object.   
We could have captured the precondition of the synchronisation being allowed
within |matching|, and dispensed with the |requires| check, as follows:
%
\begin{scala}
  def matching(spec: Spec): PartialFunction[(Sync,Sync), (Spec,(Any,Any))] = {
    case (Sync(a), Sync(b)) if !spec.bitMap(a)(b) && !spec.bitMap(b)(a) => 
      spec.sync(a, b) 
  }
\end{scala}

Most of the rest of the definitions are straightforard.  In the construction
of the |BinaryStatefulTester|, we start with a specification object whose
bitmap records no previous synchronisations.  We choose to employ the ASAP
partial order reduction.

%%%%%%%%%%%%%%%%%%%%%%%%%%%%%%%%%%%%%%%%%%%%%%%%%%%%%%%

Testing for general stateful synchronisation objects is encapsulated in the
class |StatefulTester|, described in Figure~\ref{fig:stateful-testers}.
The type parameters and most of the parameters are as for
|Binary|\-|Stateful|\-|Tester|.  The type of |matching| is adapted to capture
that a \emph{list} of invocations synchronise, as in |StatelessTester|; and
the parameter |arities| records the list of arities of synchronisations, again
as in |StatelessTester|.  The optional parameter |suffixMatching| should give
true if its argument is a non-empty suffix of a list of invocations that could
synchronise; this can be useful to optimise testers, as we describe later.

%%%%%%%%%%

As an example, a tester for a closeable channel is given in
Figures~\ref{fig:closable-chan-1} and~\ref{fig:closable-chan-2}.  A closeable
channel has an operation |close| to close the channel.  An attempt to send or
receive after the channel has been closed should fail, and throw a
|ClosedException|.  The closeable channel mixes binary and unary
synchronisations, so we cannot use |BinaryStatefulTester| here.

The testing algorithm cannot deal directly with the exceptions, so instead we
build wrappers round the operations, to map the results to proper values: the
function |trySend| maps the result of a send to a boolean, with |true|
representing success, and |false| representing failure; the function
|tryReceive| maps the result of a receive to an |Option| value, with |Some(x)|
representing the receipt of~|x|, and |None| representing a failure.

The channel has two states, open and closed.  We represent this state using a
boolean, with |true| representing that the channel has been closed.  The
function |matching| then specifies allowed synchronisations: a send and
receive can synchronise in the normal way if the channel is not closed (and
the channel remains not closed); a send or receive can fail if the channel is
closed (and the channel remains closed); and a close operation can always
succeed (even if the channel is already closed), and subsequently the channel
is closed.

%%%%%%%%%%


  %% /** Does ops represent a suffix of a possible synchronisation? */
  %% def suffixMatching(ops: List[Op]) = 
  %%   ops.length == 1 || 
  %%     (ops match{ case(List(Send(_), Receive)) => true; case _ => false })
%%%%%%%%%%

The definition of a worker is straightforward.  On each iteration, a worker
closes the channel with probability~$0.05$.  Otherwise, workers act much as
for a standard channel, but we use the functions |trySend| and |tryReceive| as
above.

We use the default value for |suffixMatching|, which treats all lists as being
a possible suffix.  We choose not to use the ASAP optimisation here, since it
seems not to help.  The rest of the definitions are then straightforward.

%%%%%%%%%%%%%%%%%%%%%%%%%%%%%%%%%%%%%%%%%%%%%%%%%%%%%%%


As another example, a testing script for a resignable barrier is in
Figures~\ref{fig:resignable-barrier-1}--\ref{fig:resignable-barrier-3}.
Recall that this is like a normal barrier, except workers may enrol or resign
from the barrier, and each barrier synchronisation is between the workers
currently enrolled.  


%%%%%%%%%%

%%%%%%%%%%

The synchronisation specification object |Spec| is parameterised by the
(immutable) set |enrolled| of identities of threads currently enrolled.  The
methods |enrol| and |resign| on |Spec| correspond to the operations with the
same names on the barriers; each definition is straightforward; the assertions
are just sanity checks, that the workers have satisfied the specification.

We represent a barrier synchronisation by a list of |Sync| objects.  As a
state-space reduction strategy, we require this list to be ordered by the
workers' identities.  The helper method |getSyncs| returns the list that would
correspond to a correct barrier synchronisation in the current state.  The
|sync| method assumes such a correct synchronisation, and gives the expected
results. 

Note that equality over |Spec| objects corresponds to value equality over
|enrolled| parameters, as required.  

The |matching| method is then straightforward. 

The |suffixMatching| function tests whether its argument is a possible suffix
of a correct synchronisation, i.e.~it is ordered by the workers' identities.
Using this function allows the underlying algorithm to avoid building
unordered lists, and leads to a fairly large speed-up, particularly for larger
numbers of threads 


|worker|

........................

Enrolled workers attempt sync with probability $0.7$. 

ASAP seems slower


\subsection{Interpreting errors}

We now discuss how to interpret the output from a tester when an error is
found\footnote{An error like this \texttt{scala synchronisationTester.ChanTester --faulty --iters 2 -p 2}}

\begin{verbatim}
0:   Call of Send(2)
0:   Return of () from Send(2)
1:   Call of Receive
2:   Call of Send(61)
1:   Return of 2 from Receive
2:   Return of () from Send(61)
3:   Call of Receive
3:   Return of 61 from Receive
Invocation 0 does not synchronise with any other 
completed operation.
\end{verbatim}

\appendix

\section{Example input scripts}
\label{app:examples}

In this appendix we give example input scripts referred to in the body of the
manual.  


\begin{figure}
\begin{scala}
object ExchangerTester extends Tester{
  /** Representation of operations within the log. */
  case class Exchange(x: Int)

  /** Specification object. */
  object Spec{
    def sync(x: Int, y: Int) = (y, x)
  }

  /** Mapping showing how synchronisations of concrete operations correspond 
    * to operations of the specification object. */
  def matching: PartialFunction[(Exchange,Exchange), (Any,Any)] = {
    case (Exchange(x), Exchange(y)) => Spec.sync(x, y) 
  }

  /** A worker.  Each worker performs a single invocation, to avoid
    * deadlocks.  */
  def worker(exchanger: ExchangerT[Int])(me: Int, log: HistoryLog[Exchange]) = {
    val x = Random.nextInt(100)
    log(me, exchanger.exchange(x), Exchange(x))
  }

  /** Do a single test. */
  def doTest = {
    val exchanger: ExchangerT[Int] = new Exchanger[Int]
    val tester = new HomogeneousBinaryStatelessTester[Exchange](
      worker(exchanger), 20, matching)
    tester()
  }

  def main(args: Array[String]): Unit = runTests(5000)
}
\end{scala}
\caption{A testing script for an exchanger, illustrating the {\scalashape
    HomogeneousBinaryStatelessTester}.\label{fig:homo-binary}} 
\end{figure}

%%%%%%%%%%%%%%%%%%%%%%%%%%%%%%%%%%%%%%%%%%%%%%%%%%%%%%%%%%%%%%%%%

\begin{figure}
\begin{scala}
object ABCTester extends Tester{
  // Representation of operations within the log
  trait Op 
  case class SyncA(id: Int) extends Op
  case class SyncB(id: Int) extends Op
  case class SyncC(id: Int) extends Op

  // The result type of an invocation.
  type IntPair = (Int,Int)

  /** The specification class. */
  object Spec{
    // Each of a, b, c get the identities of the other two
    def sync(a: Int, b: Int, c: Int) = List((b,c), (a,c), (a,b))
  }

  /** Mapping showing how synchronisations of concrete operations correspond 
    * to operations of the specification object. */
  def matching: PartialFunction[List[Op], List[IntPair]] = {
    case List(SyncA(a), SyncB(b), SyncC(c)) => Spec.sync(a, b, c) 
  }

  /** A worker with identity me. */
  def worker(abc: ABCT[Int,Int,Int])(me: Int, log: HistoryLog[Op]) = {
    for(i <- 0 until 20){
      if(me%3 == 0) log(me, abc.syncA(me), SyncA(me)) 
      else if(me%3 == 1) log(me, abc.syncB(me), SyncB(me)) 
      else log(me, abc.syncC(me), SyncC(me))
    }
  }

  def doTest = {
    val abc: ABCT[Int,Int,Int] = new ABC[Int,Int,Int]
    val tester = new StatelessTester[Op](worker(abc), 6, List(3), matching)
    tester()
  }

  def main(args: Array[String]) = runTests(1000)
}
\end{scala}
\caption{A testing script for a the ABC problem, illustrating the {\scalashape
    Stateless\-Tester}.\label{fig:ABC}} 
\end{figure}

%%%%%%%%%%%%%%%%%%%%%%%%%%%%%%%%%%%%%%%%%%%%%%%%%%%%%%%


\begin{figure}
\begin{scala}
object TimeoutChannelTester extends Tester{
  /** Representation of an operation in the log. */
  trait Op
  case class Send(x: Int) extends Op
  case object Receive extends Op

  /** Mapping showing how synchronisations of concrete operations correspond 
    * to operations of the specification object.   */
  def matching: PartialFunction[List[Op], List[Any]] = {
    case List(Send(x)) => List(false)
    case List(Receive) => List(None)
    case List(Send(x), Receive) => List(true, Some(x))
  }

  /** A worker. */
  def worker(chan: TimeoutChannelT[Int])(me: Int, log: HistoryLog[Op]) = {
    for(i <- 0 until iters){
      // Delay to ensure a mix of successul and unsuccessful invocations.
      Thread.sleep(Random.nextInt(1))
      if(Random.nextInt(2) == 0){
        val x = Random.nextInt(20)
        log(me, chan.sendWithin(x, 1+Random.nextInt(1)), Send(x))
      }
      else // receive
        log(me, chan.receiveWithin(1+Random.nextInt(1), Receive)
    }
  }

  /** Run a single test. */
  def doTest = {
    val chan: TimeoutChannelT[Int] = new TimeoutChannel[Int]
    val tester = new StatelessTester[Op](worker(chan), 4, List(1,2), matching)
    tester()
  }

  def main(args: Array[String]): Unit = runTests(1000) 
}
\end{scala}
\caption{A testing script for a timeout channel, illustrating the {\scalashape
    Stateless\-Tester}.\label{fig:timeout-chan}} 
\end{figure}


%%%%%%%%%%%%%%%%%%%%%%%%%%%%%%%%%%%%%%%%%%%%%%%%%%%%%%%

\begin{figure}
\begin{scala}
object BarrierTester extends Tester{
  /** The number of threads involved in each synchronisation. */
  var n = 4
  
  /** Representation of an operation in the log. */
  case class Sync(id: Int)

  /** Is syncs sorted by id's? */
  def isSorted(syncs: List[Sync]): Boolean = 
    syncs.length <= 1 || syncs(0).id < syncs(1).id && isSorted(syncs.tail)

  /** Mapping showing how synchronisations of concrete operations correspond 
    * to operations of the specification object. Any n invocations can
    * synchronise, and all should receive the unit value.  We require the id's to  
    * be in increasing order, to reduce the number of cases by a factor of n!. */
  def matching: PartialFunction[List[Sync], List[Unit]] = {
    case syncs if syncs.length == n && isSorted(syncs) => List.fill(n)(())
  }

  /** A worker, which calls barrier.sync. */
  def worker(barrier: BarrierT)(me: Int, log: HistoryLog[Sync]) = {
    for(i <- 0 until 20) log(me, barrier.sync(me), Sync(me))
  }

  /** Run a single test. */
  def doTest = {
    val barrier: BarrierT[Int] = new Barrier(n) 
    val tester = new StatelessTester[Sync](
      worker(barrier), n, List(n), matching, isSorted, false)
    tester()
  }

  def main(args: Array[String]) = runTests(1000)
}
\end{scala}
\caption{A testing script for a barrier synchronisation object, illustrating
  the {\scalashape Stateless\-Tester}.\label{fig:barrier}}
\end{figure}


\begin{figure}
\begin{scala}
object OneFamilyTester extends Tester{
  /** Number of threads to run. */
  var n = 4

  /** Representation of operations within the log. */
  case class Sync(id: Int)

  type BitMap = Array[Array[Boolean]]

  /** The specification class.  bitMap shows which threads have already
    * synchronised. */
  class Spec(val bitMap: BitMap){
    def sync(a: Int, b: Int): (Spec, (Int, Int)) = {  
      // These two must not have sync'ed before
      require(!bitMap(a)(b) && !bitMap(b)(a))
      val newBitMap = bitMap.map(_.clone)      // Create updated bitmap.
      newBitMap(a)(b) = true; newBitMap(b)(a) = true
      (new Spec(newBitMap), (b,a)) 
    }

    override def equals(that: Any) = that match{
      case s: Spec => 
        (0 until n).forall(a => bitMap(a).sameElements(s.bitMap(a)))
    }

    override def hashCode = {
      var h = 0
      for(a <- 0 until n; b <- 0 until n){ 
        h = h << 1; if(bitMap(a)(b)) h += 1 
      }
      h
    }
  } // end of Spec

  // ...
}
\end{scala}
\caption{A testing script for a the one family problem, illustrating the
  {\scalashape Binary\-Stateful\-Tester} (part~1).\label{fig:one-family-1}}
\end{figure}

%%%%%

\begin{figure}
\begin{scala}
  /** Mapping showing how synchronisations of concrete operations correspond 
    * to operations of the specification object. */
  def matching(spec: Spec): PartialFunction[(Sync,Sync), (Spec,(Any,Any))] = {
    case (Sync(a), Sync(b)) => spec.sync(a, b) 
  }

  /** A worker.  */
  def worker(of: OneFamilyT)(me: Int, log: HistoryLog[Sync]) = {
    for(_ <- 0 until n-1) log(me, of.sync(me), Sync(me))
  }

  /** Do a single test. */
  def doTest = {
    val of: OneFamilyT = new OneFamily(n)
    val spec = new Spec(Array.ofDim[Boolean](n,n))
    val bst = new BinaryStatefulTester[Sync,Spec](
      worker(of), n, matching, spec, true)
    bst()
    }
  }

  def main(args: Array[String]) = runTests(5000)
\end{scala}
\caption{A testing script for a the one-family problem, illustrating the
  {\scalashape Binary\-Stateful\-Tester} (part~2).\label{fig:one-family-2}}
\end{figure}

%%%%%%%%%%%%%%%%%%%%%%%%%%%%%%%%%%%%%%%%%%%%%%%%%%%%%%%
\begin{figure}
\begin{scala}
object CloseableChanTester extends Tester{
  /** Representation of an operation in the log. */
  trait Op
  case class Send(x: Int) extends Op
  case object Receive extends Op
  case object Close extends Op

  /** Try to send x on chan, catching a ClosedException.  Return true if
    * successful. */ 
  @inline private def trySend(chan: CloseableChan[Int], x: Int): Boolean = 
    try{ chan.send(x); true } catch { case _: ClosedException => false }

  /** Try to receive on chan, catching a ClosedException.  Optionally return the
    * value received. */ 
  @inline private def tryReceive(chan: CloseableChan[Int]): Option[Int] = 
    try{ Some(chan.receive()) } catch { case _: ClosedException => None }

  /** Mapping showing how synchronisations of concrete operations correspond 
    * to operations of the specification object.  Here the specification object
    * is simply a Boolean, indicating whether the channel is closed.  */
  def matching(closed: Boolean): PartialFunction[List[Op], (Boolean, List[Any])] = {
    case List(Send(x), Receive) if !closed => (closed, List(true, Some(x)))
    case List(Send(x)) if closed => (closed, List(false))
    case List(Receive) if closed => (closed, List(None))
    case List(Close) => (true, List(()))
  }

  // ...
}
\end{scala}
\caption{A testing script for a closeable channel, illustrating the {\scalashape
    Stateful\-Tester} (part~1).\label{fig:closable-chan-1}} 
\end{figure}

%%%%%

\begin{figure}
\begin{scala}
  /** A worker.  Workers close the channel with probability 0.05;
    * otherwise, workers with an even identity send; workers with an odd
    * identity receive.  */
  def worker(chan: CloseableChan[Int])(me: Int, log: HistoryLog[Op]) = {
    for(i <- 0 until iters){
      if(Random.nextFloat() < 0.05) log(me, chan.close, Close)
      else if(me%2 == 0){
        val x = Random.nextInt(MaxVal); log(me, trySend(chan, x), Send(x))
      }
      else log(me, tryReceive(chan), Receive)
    }
  }    

  /** Run a single test. */
  def doTest = {
    val chan: CloseableChan[Int] = new CloseableSyncChan[Int]
    val tester = new StatefulTester[Op,Boolean](
      worker(chan), 4, List(1,2), matching, spec0 = true, doASAP = false)
    tester()
  }

  def main(args: Array[String]) = runTests(5000)
\end{scala}
\caption{A testing script for a closeable channel, illustrating the {\scalashape
    Stateful\-Tester} (part~2).\label{fig:closable-chan-2}} 
\end{figure}

%%%%%%%%%%%%%%%%%%%%%%%%%%%%%%%%%%%%%%%%%%%%%%%%%%%%%%%

\begin{figure}
\begin{scala}
object ResignableBarrierTester{
  /** The type of barriers. */
  type Barrier = ResignableBarrierT[Int]

  /** Operations. */
  abstract class Op
  case class Enrol(id: Int) extends Op
  case class Resign(id: Int) extends Op
  case class Sync(id: Int) extends Op

  /** Type of the set of threads currently enrolled. */
  type Enrolled = HashSet[Int]

  /** The specification class.
    * @param enrolled the set of threads currently enrolled.  */
  case class Spec(enrolled: Enrolled){
    /** The effect of an enrol invocation. */
    def enrol(id: Int) = { 
      assert(!enrolled.contains(id)); (new Spec(enrolled + id), List(()))
    }

    /** The effect of a resign invocation. */
    def resign(id: Int) = {
      assert(enrolled.contains(id)); (new Spec(enrolled - id), List(()))
    }

    /** The list of Sync objects that would correspond to a barrier
      * synchronisation in the current state. */
    def getSyncs: List[Sync] = enrolled.toList.sorted.map(Sync)

    /** The effect of a barrier synchronisation.  Pre: syncs = getSyncs.  */
    def sync(syncs: List[Op]) = {
      val n = enrolled.size; assert(syncs.length == n)
      (this, List.fill(n)(()))
    }
  } // end of Spec

  ...
}
\end{scala}
\caption{A testing script for a resignable barrier, illustrating the
  {\scalashape Stateful\-Tester} (part~1).  \label{fig:resignable-barrier-1}}
\end{figure}

%%%%%%%%%%

\begin{figure}
\begin{scala}
  /** Partial function showing how a list of invocations can synchronise, and
    * returning the expected next state and list of return values. */ 
  def matching(spec: Spec): PartialFunction[List[Op], (Spec, List[Unit])] = {
    ops => ops match{
      case List(Enrol(id)) => spec.enrol(id)
      case List(Resign(id)) => spec.resign(id)
      case syncs if syncs == spec.getSyncs => spec.sync(syncs)
        // Note: the above "if" clause tests the precondition for this being a
        // valid barrier synchronisation.
    }
  }

  /** Does `ops` represent a suffix of a possible synchronisation (including the
    * unary operations)? */
  def suffixMatching(ops: List[Op]) = ops.length <= 1 || suffixMatching1(ops, 0)
      
  /** Is ops a list of Sync values, with increasing values of id, all at least
    * n? */
  def suffixMatching1(ops: List[Op], n: Int): Boolean = 
    if(ops.isEmpty) true
    else ops.head match{
      case Sync(m) if m >= n => suffixMatching1(ops.tail, m+1)
      case _ => false
    }

  /** A worker. */
  def worker(barrier: Barrier)(me: Int, log: HistoryLog[Op]) = {
    var enrolled = false
    for(i <- 0 until 4){
      if(enrolled){
        if(Random.nextFloat() < 0.7) log(me, barrier.sync(me), Sync(me))
        else{ log(me, barrier.resign(me), Resign(me)); enrolled = false }
      }
      else{ log(me, barrier.enrol(me), Enrol(me)); enrolled = true }
    }
    // Resign at the end, to avoid deadlocks
    if(enrolled) log(me, barrier.resign(me), Resign(me))
  }
\end{scala}
\caption{A testing script for a resignable barrier, illustrating the
  {\scalashape Stateful\-Tester} (part~2).  \label{fig:resignable-barrier-2}}
\end{figure}


%%%%%%%%%%

\begin{figure}
\begin{scala}
  var p = 4 // # workers

  /** Do a single test. */
  def doTest() = {
    val barrier = new ResignableBarrier[Int](faulty)
    val spec = new Spec(new Enrolled)
    val tester = new StatefulTester[Op,Spec](
      worker(barrier), p, (1 to p).toList, matching, suffixMatching, spec, false)
    if(!tester()) sys.exit()
  }

  def main(args: Array[String]) = runTests(10000)
}
\end{scala}
\caption{A testing script for a resignable barrier, illustrating the
  {\scalashape Stateful\-Tester} (part~3).  \label{fig:resignable-barrier-3}}
\end{figure}


\end{document}
