\subsection{Stateful testers}

We now consider stateful testers.  

The algorithm for testing binary stateful synchronisation objects is
encapsulated in the class |BinaryStatefulTester| whose signature is in
Figure~\ref{fig:stateful-testers}.  The parameters are as follows.
%
\begin{itemize}
\item The type parameter |Op| is the representation of operation invocations
  in the log, as before.

\item The type parameter |S| is the type of synchronisation specification
  objects, giving an abtract representation of the synchronisation object.
  Such objects should be \emph{immutable}; they should have a suitable
  definition for equality (|equals|) and a compatible |hashCode|.

\item The parameter |worker| defines a worker that operates on the
  synchronisation object, as before.

\item The parameter~|p| gives the number of threads to run, as before.

\item The parameter |specMatching| captures the specification.  This function
  takes a parameter~|s| corresponding to the state of the specification
  object.  Then |specMatching(s)| is a partial function defining what
  synchronisations are allowed given the state~|s| for the specification
  object.  Its domain, as before, is the pairs of invocations that may
  synchronise; the function returns the resulting state of the specification
  object, and the results to be returned by the two invocations.

\item The parameter |spec0| is the initial state of the specification object.

\item The optional parameter |doASAP| specifies whether the ASAP partial order
  reduction should be employed.  Experience suggests that this is normally
  beneficial. 
\end{itemize}

%%%%%%%%%%

\begin{figure}
\begin{scala}
class BinaryStatefulTester[Op,S](
  worker: (Int, HistoryLog[Op]) => Unit,
  p: Int,
  specMatching: S => PartialFunction[(Op,Op), (S,(Any,Any))],
  spec0: S,
  doASAP: Boolean = false
){
  def apply(delay: Int = -1): Boolean
}

class StatefulTester[Op,S](
  worker: (Int, HistoryLog[Op]) => Unit,
  p: Int, 
  arities: List[Int],
  matching: S => PartialFunction[List[Op], (S,List[Any])],
  suffixMatching: List[Op] => Boolean = (es: List[Op]) => true,
  spec0: S, 
  doASAP: Boolean = false
){
  def apply(delay: Int = -1): Boolean
}
\end{scala}
\caption{Signatures for the stateful testers.\label{fig:stateful-testers}}
\end{figure}

%

%%%%%%%%%%%%%%%%%%%%%%%%%%%%%%%%%%%%%%%%%%%%%%%%%%%%%%%

%%%%%

Figures~\ref{fig:one-family-1} and~\ref{fig:one-family-2} give an example, for
the one-family problem.  Here, |n| threads, with identities $\upto0{\sm{n}}$,
each call a method~|sync| at most $\sm{n}-1$ times, passing in its own
identity.  Each such invocation should synchronise with another invocation,
and return the identity of the other thread; however, each pair of threads
should synchronise together at most once.  Hence the synchronistion object is
stateful: abstractly, its state is the set of pairs of threads that have
synchronised so far.

The specification object |Spec(bitMap)| captures this state using the
bitmap~|bitMap|; for each pair of threads~|a| and |b|,\, |bitMap(a)(b)| is
true if they have already synchronised.  The operation |sync(a, b)| specifies
the result of a synchronisation between threads~|a| and~|b|.  This is allowed
only if |a| and |b| have not already synchronised, as captured by the
|requires| check; if this check fails, the testing framework catches the
resulting |Illegal|\-|Argument|\-|Exception|, but does not allow the
synchronisation to be linearised in this state.  If the synchronisation is
allowed, it creates a new bitmap recording the synchronisation, and returns a
corresponding new specification object, together with the correct results for
the two invocations.  Recall that the specification object must be immutable:
hence we create a new specification object rather than simply updating the
current one.  Recall also that the specification object must have appropriate
definitions of equality and hash code: we define equality as value equality
over the bitmaps, and the hash code based directly on the content of the
bitmap.

The |matching| function defines that two invocations may synchronise as
captured by the |sync| method on the current specification object.   
We could have captured the precondition of the synchronisation being allowed
within |matching|, and dispensed with the |requires| check, as follows:
%
\begin{scala}
  def matching(spec: Spec): PartialFunction[(Sync,Sync), (Spec,(Any,Any))] = {
    case (Sync(a), Sync(b)) if !spec.bitMap(a)(b) && !spec.bitMap(b)(a) => 
      spec.sync(a, b) 
  }
\end{scala}

Most of the rest of the definitions are straightforard.  In the construction
of the |BinaryStatefulTester|, we start with a specification object whose
bitmap records no previous synchronisations.  We choose to employ the ASAP
partial order reduction.

%%%%%%%%%%%%%%%%%%%%%%%%%%%%%%%%%%%%%%%%%%%%%%%%%%%%%%%

Testing for general stateful synchronisation objects is encapsulated in the
class |StatefulTester|, described in Figure~\ref{fig:stateful-testers}.
The type parameters and most of the parameters are as for
|Binary|\-|Stateful|\-|Tester|.  The type of |matching| is adapted to capture
that a \emph{list} of invocations synchronise, as in |StatelessTester|; and
the parameter |arities| records the list of arities of synchronisations, again
as in |StatelessTester|.  The optional parameter |suffixMatching| should give
true if its argument is a non-empty suffix of a list of invocations that could
synchronise; this can be useful to optimise testers, as we describe later.

%%%%%%%%%%

As an example, a tester for a closeable channel is given in
Figures~\ref{fig:closable-chan-1} and~\ref{fig:closable-chan-2}.  A closeable
channel has an operation |close| to close the channel.  An attempt to send or
receive after the channel has been closed should fail, and throw a
|ClosedException|.  The closeable channel mixes binary and unary
synchronisations, so we cannot use |BinaryStatefulTester| here.

The testing algorithm cannot deal directly with the exceptions, so instead we
build wrappers round the operations, to map the results to proper values: the
function |trySend| maps the result of a send to a boolean, with |true|
representing success, and |false| representing failure; the function
|tryReceive| maps the result of a receive to an |Option| value, with |Some(x)|
representing the receipt of~|x|, and |None| representing a failure.

The channel has two states, open and closed.  We represent this state using a
boolean, with |true| representing that the channel has been closed.  The
function |matching| then specifies allowed synchronisations: a send and
receive can synchronise in the normal way if the channel is not closed (and
the channel remains not closed); a send or receive can fail if the channel is
closed (and the channel remains closed); and a close operation can always
succeed (even if the channel is already closed), and subsequently the channel
is closed.

%%%%%%%%%%


  %% /** Does ops represent a suffix of a possible synchronisation? */
  %% def suffixMatching(ops: List[Op]) = 
  %%   ops.length == 1 || 
  %%     (ops match{ case(List(Send(_), Receive)) => true; case _ => false })
%%%%%%%%%%

The definition of a worker is straightforward.  On each iteration, a worker
closes the channel with probability~$0.05$.  Otherwise, workers act much as
for a standard channel, but we use the functions |trySend| and |tryReceive| as
above.

We use the default value for |suffixMatching|, which treats all lists as being
a possible suffix.  We choose not to use the ASAP optimisation here, since it
seems not to help.  The rest of the definitions are then straightforward.

%%%%%%%%%%%%%%%%%%%%%%%%%%%%%%%%%%%%%%%%%%%%%%%%%%%%%%%


As another example, a testing script for a resignable barrier is in
Figures~\ref{fig:resignable-barrier-1}--\ref{fig:resignable-barrier-3}.
Recall that this is like a normal barrier, except workers may enrol or resign
from the barrier, and each barrier synchronisation is between the workers
currently enrolled.  


%%%%%%%%%%

%%%%%%%%%%

The synchronisation specification object |Spec| is parameterised by the
(immutable) set |enrolled| of identities of threads currently enrolled.  The
methods |enrol| and |resign| on |Spec| correspond to the operations with the
same names on the barriers; each definition is straightforward; the assertions
are just sanity checks, that the workers have satisfied the specification.

We represent a barrier synchronisation by a list of |Sync| objects.  As a
state-space reduction strategy, we require this list to be ordered by the
workers' identities.  The helper method |getSyncs| returns the list that would
correspond to a correct barrier synchronisation in the current state.  The
|sync| method assumes such a correct synchronisation, and gives the expected
results. 

Note that equality over |Spec| objects corresponds to value equality over
|enrolled| parameters, as required.  

The |matching| method is then straightforward. 

The |suffixMatching| function tests whether its argument is a possible suffix
of a correct synchronisation, i.e.~it is ordered by the workers' identities.
Using this function allows the underlying algorithm to avoid building
unordered lists, and leads to a fairly large speed-up, particularly for larger
numbers of threads 


|worker|

........................

Enrolled workers attempt sync with probability $0.7$. 

ASAP seems slower
