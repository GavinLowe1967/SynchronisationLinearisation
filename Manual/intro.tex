
\section{Introduction}

This manual describes a package for testing synchronisation objects.  The
package uses the Scala programming language~\cite{programming-in-Scala}.  We
briefly describe the purpose of such synchronisation objects, and appropriate
correctness conditions.  For full details, see~\cite{sync}.  

A \emph{synchronisation object} allows synchronisations between two or more
threads.  The most common form of synchronisation object is a synchronous
channel.  Here, threads call operations to send or receive data.  Each
invocation should overlap in time with an invocation of the opposite sort, and
the receive operation should return the value sent by the send operation.

More generally, a synchronisation involves some number of invocations that
overlap in time and that (normally) exchange data.  The specification for the
object describes what synchronisations are allowed (i.e.~which operations
synchronise with which others), and the values each should return. 

We use the term \emph{binary synchronisation} for a synchronisation between
two invocations; and we use the term \emph{binary synchronisation object} for
a synchronisation object all of whose synchronisations are binary. 

Synchronisations in a synchronous channel are between two \emph{different}
operations, sending and receiving; we call such synchronisations
\emph{heterogeneous}.  By contrast, some synchronisations involve two
invocations of the \emph{same} operation; we call such synchronisations
\emph{homogeneous}.  For example, an exchanger object has a single operation,
|exchange|, taking a single parameter; two invocations of this operation
should synchronise, and each should return the other's parameter, so the two
threads exchange data values.

Some synchronisations are between more than two threads.  We use the term
\emph{arity} for the number of threads involved in a synchronisation.  For
example, a barrier synchronisation object allows synchronisations between
|n|~threads, where |n| is a parameter of the object's constructor; so these
synchronisations have arity~|n|.  

Further, we can consider an invocation that synchronises with no other
invocation as a \emph{unary} synchronisation (arity~1).  Some operations might
have mixed modes of synchronisations, maybe with different arities; for
example, a channel with a timeout might perform a successful binary
synchronisation, or might fail to synchronise with another thread and timeout,
and so have a unary synchronisation.

Synchronisation objects may maintain some state between one synchronisation
and another.  For example, a closeable channel is like a synchronous channel,
except it can be closed, after which any attempt to send or receive will fail;
hence it has two states, open and closed.  As another example, a resignable
barrier is like a normal barrier, except threads may enrol and resign from the
barrier; each synchronisation is between all the currently enrolled threads;
hence the state of the object is (abstractly) the set of currently enrolled
threads.  We use the term \emph{stateful} for such synchronisation objects,
and use the term \emph{stateless} otherwise. 

The standard correctness condition for synchronisation objects is
\emph{synchronisation linearisation}~\cite{sync}.  This states that the
synchronisations appear to take place in a one-at-a-time order, with each
synchronisation happening between the calls and returns of the relevant
invocations; further, the values returned by each call should match an
appropriate specification for the object (which might depend on the previous
synchronisations in the case of a stateful synchronisation object).  This is a
safety property.

The liveness property \emph{synchronisation progressibility} states, in
addition, that invocations don't get stuck unnecessarily: if there is a
possible synchronisation between pending invocations, then some such
synchronisation should happen; and once a synchronisation has occurred, the
relevant invocations should return.

%%%%%%%%%%

\subsection{Testing synchronisation objects}

Our approach to testing synchronisation objects is as follows.  We run a
number of \emph{worker threads}, that invoke operations upon the object,
logging the calls and returns.  We then run an algorithm over the log history
to test whether it is synchronisation linearisable and (optionally)
synchronisation progressible.  Appropriate algorithms are described
in~\cite{sync}: the testing framework encapsulates these algorithms, and it is
not necessary for the reader to understand them in order to use the framework.
However, different algorithms are appropriate in different cases, depending
on: whether synchronisations are binary or not; in the binary case, whether
synchronisations are heterogeneous or homogeneous; and whether the
synchronisation object is stateful or stateless.

In the remainder of this manual, we describe how to use the testing framework
for carrying out such tests.  In Section~\ref{sec:example}, we present a
simple input script, describing the technique for binary heterogeneous
stateless objects: we describe how to test for synchronisation linearisation;
describe various pragmatics; and then describe how to test for synchronisation
progressibility.  In Section~\ref{sec:algorithms} we describe how to use the
other testing algorithms, for homegeneous, non-binary or stateful
synchronisation objects.   In Section~\ref{sec:errors} we describe how to
interpret the output produced by a tester when it finds a history that is not
synchronisation linearisable, or not synchronisation progressibility: the
tester prints the history together with some explanation.  \framebox{\ldots}


We assume familiarity with Scala~\cite{programming-in-Scala} throughout this
manual.


\framebox{Installation and usage}: Appendix?
