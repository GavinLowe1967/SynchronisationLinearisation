\section{Other algorithms}
\label{sec:algorithms}

In the previous section, we considered the class |BinaryStatelessTester| for
testing binary heterogeneous stateless synchronisation objects.  In this
section, we describe classes that encapsulate other testing algorithms.

All example scripts referred to in this section can be found in
Appendix~\ref{app:examples}.  Fuller versions of each script are included in
the distribution.

Each script tests only for synchronisation linearisation, but can be adapted
to test for progressibility by passing a positive integer to the |apply|
method of the object encapsulating the algorithm, as for the synchronous
channel tester in the previous section.

%%%%%%%%%%%%%%%%%%%%%%%%%%%%%%%%%%%%%%%%%%%%%%%%%%%%%%%

\subsection{Stateless testers}

Figure~\ref{fig:stateless-testers} gives the signatures for each of the
classes that encapsulates an algorithm for testing a stateless synchronisation
object (including |BinaryStatelessTester|, described in the previous section).

The class |HomogeneousBinaryStatelessTester| is for testing binary homogeneous
stateless synchronisation objects.  It takes the same parameters as in the
heterogeneous case; and likewise the |apply| function takes the same optional
parameter.

%%%%%%%%%%

\begin{figure}
\begin{scala}
/** Testing algorithm for binary heterogeneous stateless synchronisation
  * objects. */
class BinaryStatelessTester[Op](
    worker: (Int, HistoryLog[Op]) => Unit,
    p: Int,
    matching: PartialFunction[(Op,Op), (Any,Any)]
){
  def apply(delay: Int = -1): Boolean
}

/** Testing algorithm for binary homogeneous stateless synchronisation
  * objects. */
class HomogeneousBinaryStatelessTester[Op](
    worker: (Int, HistoryLog[Op]) => Unit,
    p: Int,
    matching: PartialFunction[(Op,Op), (Any,Any)]
){
  def apply(delay: Int = -1): Boolean
}

/** Testing algorithm for general binary stateless synchronisation objects. */
class StatelessTester[Op](  
    worker: (Int, HistoryLog[Op]) => Unit,
    p: Int, 
    arities: List[Int],
    matching: PartialFunction[List[Op], List[Any]],
    suffixMatching: List[Op] => Boolean = (es: List[Op]) => true
){
  def apply(delay: Int = -1): Boolean
}
\end{scala}
\caption{Signatures for the stateless testers.\label{fig:stateless-testers}}
\end{figure}

%%%%%%%%%%

Figure~\ref{fig:homo-binary} gives an example for this class, giving a testing
script for an exchanger.  Threads call the method |exchange| on the exchanger,
passing in a value; this call should synchronise with another call, and both
threads should receive the other's value.  

Most parts of the script are straightforward, and similar to the script for
the synchronous channel.  

One point to note about this script is that each worker thread performs a
\emph{single} invocation: otherwise it is possible for the system to deadlock,
for example if one thread has two remaining invocations but all the others
have terminated.  This tactic might be of use elsewhere.
 
%%%%%%%%%%%%%%%%%%%%%%%%%%%%%%%%%%%%%%%%%%%%%%%%%%%%%%%%%%%%%%%%%

The class |StatelessTester| can test arbitrary stateless synchronisation
objects.  Its signature is again in Figure~\ref{fig:stateless-testers}.
The parameters |Op|, |worker| and~|p| are as for the binary testers.  The
parameter |arities| is a list of all possible arities of synchronisations.
The parameter, |matching| is much like in the binary case, except its domain
is all \emph{lists} of operations that might synchronise, and it returns a
corresponding \emph{list} of expected results.  Finally, the optional
parameter |suffixMatching| is a function that should return |true| when its
argument is a suffix of a possible synchronisation (with default value that
always returns |true|); we explain below how this can be useful in
optimisations.

%%%%%

Figure~\ref{fig:ABC} gives an example, giving a testing script for the ABC
problem.  Here, the synchronisation object provides three operations,
|syncA(a: A)|, |syncB(b: B)|, and |syncC(c: C)|.  Each synchronisation should
be between three invocations, one of each operation, with each invocation
returning the parameters of the other two invocations.

Most aspects of the tester are straightforward.  Here the tester runs 6
threads (the number must be a multiple of~3 to avoid deadlocks), with two for
each operation.  Each synchronisation involves 3~invocations.

%%%%%

Figure~\ref{fig:timeout-chan} gives another example, for a timeout channel,
where an invocation can fail to synchronise and timeout.  The |sendWithin|
operation returns a boolean, indicating whether is correctly sent its value.
The |receiveWithin| operation returns an |Option| value, with |Some(x)|
indicating that it received~|x|, and |None| indicating that it timed out.

The definition of |matching| illustrates how to specify mixed modes of
synchronisation: a |send(x)| may fail to synchronise, so timeout and return
|false|; a |receive| may fail to synchronise, so timeout and return |None|; or
a |send(x)| and |receive| may synchronise and return |true| and |Some(x)|,
respectively.  Thus synchronisations may have arities~1 or~2, as captured by
the parameter |List(1,2)| of the |StatelessTester| constructor.

%%%%%%%%%%

Figure~\ref{fig:barrier}  gives an example of a
tester for a barrier synchronisation object.  Each such object is used by some
number~|n| of threads, each of which calls an operation~|sync|: no call to
|sync| should return until all~|n| threads have called it, so this
synchronises all |n|~threads.

Most aspects of the script are straightforward.  A synchronisation will be
represented by a list of |n| |Sync| objects, one for each worker.  However,
with a naive approach, each such synchronisation could be represented in
$\sm{n}!$ different ways, giving an increase in the complexity of checking.
We therefore make the decision that we will require each such list to be in
sorted order of the workers' identities (as tested by the recursive |isSorted|
function), so as to reduce the number of cases considered. 

We also supply an argument for the optional parameter |suffixMatching| of the
|StatelessTester| constructor.  Recall that this parameter tests whether its
argument is a suffix of a possible synchronisation, so this function tests if
the identities are of the form $\interval{k}{\sm n}$ for some~$k$.
Internally to the |StatelessTester|, this reduces the
number of lists of operations built as possible synchronisations.  When run
with six worker threads, the use of this parameter reduces the running time of
the tester by a factor if over 20, although the speed-up is less with fewer
workers.

